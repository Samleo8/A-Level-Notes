\documentclass[Physics.tex]{subfiles}
\begin{document}
\chapter{Dynamics}
\section{Newton's laws of motion}
\sldef{Newton's first law} of motion states that a body will continue in its state of rest or uniform motion in a straight line unless an external resultant force acts on it.

\sldef{Newton's second law} of motion states that the rate of change of linear momentum of a body is directly proportional to the resultant force acting on it and the change takes place in the direction of the force.

\sldef{Newton's third law} of motion states that if object A exerts a force on object B, then object B exerts a force of equal magnitude but in the opposite direction on object A.
\section{Momentum}
The \sldef{linear momentum} \(\mathbf{p}\) of an object is the product of its mass and velocity. \begin{equation}\mathbf{p} = m\mathbf{v}\end{equation} The SI unit of momentum is \si{\newton\second}. The \sldef{newton} (\si{\newton}) is the force required to effect a change in linear momentum at the rate of \SI{1}{\kilogram\metre\per\square\second}. The \sldef{impulse} of a force \(\Delta\mathbf{p}\) is equal to the change in linear momentum caused by the force over some time period.

\sldef{Average force} is the impulse over total time taken. \begin{equation}\langle\mathbf{F}\rangle\ = \frac{\Delta\mathbf{p}}{\Delta t}\end{equation}

The \sldef{principle of conservation of momentum} states that the total momentum of a system of objects remains constant provided no external resultant force acts on the system.

In an \sldef{elastic collision}, both momentum and kinetic energy are conserved. In an \sldef{inelastic collision}, momentum is conserved but kinetic energy is not. In a \sldef{completely inelastic collision}, momentum is conserved but kinetic energy is not, and the objects stick together after collision.

For an elastic head-on collision of two objects, the relative speed of approach is equal to the relative speed of separation, regardless of the objects' masses. \begin{equation}\mathbf{v}_2 - \mathbf{v}_1 = \mathbf{u}_1 - \mathbf{u}_2\end{equation}
\end{document}