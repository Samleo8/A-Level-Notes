\documentclass[Economics.tex]{subfiles}
\begin{document}
\chapter{Macroeconomic Principles}
\section{Macroeconomic aims}
Governments generally have four macroeconomic aims.

Governments aim for high and sustained economic growth as it leads to a higher material standard of living for their citizens, a reduction in unemployment, an increase in tax revenue, and a fall in government expenditure on things like unemployment benefits as unemployment falls. Stable economic growth also boosts individuals' and firms' confidence, encouraging further consumption and investment.

Governments aim for low and stable inflation --- typically \SIrange{2}{3}{\percent} --- as high or uncertain inflation adversely affects investment, production, employment and economic growth. It will also distort the price mechanism, leading to allocative inefficiency. If domestic inflation is higher than competing countries, the country's export competitiveness will be impacted.

Governments aim for full employment (i.e.\ no disequilibrium unemployment) as unemployment signifies a waste of resources: the economy is not producing at its full capacity. Unemployment means lower output and income and thus a lower standard of living for the economy; it also means less tax revenue as the unemployed do not pay income tax --- and worse, the government may have to pay unemployment benefits.

Governments aim for a balance of payments equilibrium (or a small surplus) as a long term, persistent and unplanned deficit indicates a fundamental problem in the economy: that the country is living beyond its means. The government will have to draw on its foreign reserves to make up the shortfall, and these reserves will eventually deplete, after which the government will have to borrow from abroad, leading to long term debt, impacting the standard of living of future generations.
\section{Circular Flow of Income}
The circular flow of income shows the flow of payments for goods and services around the economy. Typically, a 4-sector economy is used, encompassing households, firms, the government, and the foreign sector.

The main role of households is to supply factors of production they own i.e.\ labour in return for factor payments. Conversely, firms supply goods and services to households. Thus household expenditure forms firm income, and firm expenditure forms household income.

An injection (\(J\)) is an addition to the circular flow which does not come from household expenditure; it is mainly formed by investments (\(I\)), government spending (\(G\)) and export revenue (\(X\)). A withdrawal (\(W\)) is any part of income that is not passed on within the circular flow of income; it is mainly formed by savings (\(S\)), taxes (\(T\)) and import expenditure (\(M\)).

In an economy at equilibrium, the value of injections equals the value of withdrawals i.e.\ \(J = W\). When there is an increase in injections or withdrawals, the multiplier effect takes place.
\subsection{Multiplier process}\label{s:6.mult}
Assume we have a 4-sector economy that is initially at equilibrium, where injections equals withdrawals, the marginal propensity to withdraw is \num{0.1}, and that the economy has spare capacity.

An injection into the economy of \dol[m]{1} will cause firms to respond to this increase by using more labour and raw materials to produce more goods, paying more factor incomes to households. The income of households supplying factors of production increases by \dol[m]{1}, and national income increases by \dol[m]{1}.

Since the marginal propensity to withdraw is \num{0.1}, households supplying factors of production will spend \dol[m]{0.9} on food, clothing, recreation, medical care, and the like, and withdraw \dol[m]{0.1} as savings, taxes, and import expenditure. This increase in spending by households causes firms to use more labour and raw materials to produce more output in the food, clothing, recreation, medical care, etc.\ industries, and households supplying factors of production in those industries will receive an income of \dol[m]{0.9}.

This second group of recipients will spend \dol[m]{0.81} on goods and services and withdraw the remaining \dol[m]{0.09}. This process continues, with each new round of spending being \num{0.9} of the previous. A long chain of extra income, household spending, employment and withdrawal is created. Each round of household spending becomes smaller as each time money circulates, some of it is withdrawn. Eventually, when the injection of \dol[m]{1} has leaked out as withdrawals, and the increase in withdrawals equals the increase in injections, the multiplier process ceases and there is no further change in national income. A new equilibrium is reached.

National income will increase by a multiplied amount, with the multiplier \(k = 1/\MPW = 10\), so the total increase in national income is \(10 \times \dol[m]{1}\), or \dol[m]{10}.
\subsection{Multiplier}
There are various factors affecting the multiplier \(k\), which is the reciprocal of the marginal propensity to withdraw. Clearly, if households spend less in the domestic economy, there will be a smaller multiplier effect. In other words, any change in national income will propagate through the circular flow until the initial increase in income leaks away from the inner flow as withdrawals.

The marginal propensity to withdraw is composed of the marginal propensity to save \MPS{}, marginal rate of taxation \MRT{}, and marginal propensity to import \MPM{}: \[\MPW = \MPS + \MRT + \MPM\] Each component depends on several factors.

The marginal propensity to save depends on a country's distribution of income and availability of welfare, whether the country has compulsory savings, and even on the culture of the people. In Singapore, the \MPS{} is high because of our CPF scheme, which forces us to save. An unequal distribution of income in Singapore also leads to higher \MPS{} as the higher income tend to save more. Obviously, the lack of a welfare system in Singapore also contributes to our high \MPS{}.

The marginal rate of taxation depends (quite obviously) on the tax rates.

The marginal propensity to import depends mostly on the national income of a country, its factor endowment and its openness to international trade. Singapore has a high \MPM{}, given that it is a small and open economy. Singapore also lacks natural resources, forcing it to import many of the consumer goods and raw materials from overseas. Its generally higher national income also means that its population is willing and able to spend on imports.
\section{Aggregate demand and aggregate supply}
Aggregate supply (\AS{}) is the total output that firms in an economy are willing and able to supply at different price levels in a given period of time. The aggregate supply curve is generally perfectly elastic at low levels of output, upward sloping over a small range, and then perfectly inelastic. This is because the economy in the long run can operate at any level of output that is not necessarily at full capacity.

The \AS{} curve will shift to the right if there is an increase in the productive capacity of the economy, in the quantity and/or quality of resources, or an improvement in technology.

Aggregate demand (\AD{}) is the total spending on an economy's products at different general price levels in a given time period, measured in real terms. It consists of consumption, investment, government spending, and net exports i.e. \[\AD = C + I + G + (X-M)\]

When components of \AD{} increase, \AD{} increases, and vice versa.

Consumption \(C\) generally increases when there is an increase in income or wealth, as the purchasing power of individuals increases, or a fall in interest rate, as individuals become more willing and able to borrow to purchase expensive items.

Investment \(I\) generally increases when there is a decrease in interest rate, as projects with lower expected returns become profitable, since the cost of borrowing is less, so more investment occurs. \(I\) is also affected by business confidence: when businesses are optimistic, \(I\) is usually higher, and vice versa.

Government expenditure \(G\) depends on what government policies have been set in place.

Export revenue \(X\) and import expenditure \(M\) are both affected by exchange rates of the domestic currency. When the domestic currency appreciates, \(X\) falls (assuming \(\PEDx > 0\)) and \(M\) (assuming \(\PEDm > 1\)) increases as exports become more expensive in foreign currency, so overseas consumers switch their consumption to alternative, cheaper countries, while imports become cheaper in domestic currency and so domestic buyers switch from more expensive do\-mes\-tic\-ally-pro\-duced products to cheaper imports.
\subsection{Adjustment process}
The value of aggregate demand at a given price level is the total spending, while that of aggregate supply is total output. The economy is at equilibrium when total spending equals total supply.

If total spending is less than total output, there is a surplus in the economy. Goods remain unsold and inventories accumulate, so firms lower production. If the economy is near full employment, there will be less competition for goods and factor prices fall, leading to a fall in general price level. Total spending rises while total output falls until they are equal, and real output thus falls.

If total spending is more than total output, there is a shortage in the economy. Inventories deplete and firms increase output to meet demand. If the economy is near full employment, there will be more competition for goods and so firms bid up factor prices, and general price level rises. Total spending falls while total output rises until they are equal, and real output thus increases.

If there is a shortage when the economy is at full employment, firms bid up factor prices and general price level increases until total spending equals total output, but total output does not change.

If there is a shortage when the economy has spare capacity, unused resources are simply hired at the same costs, there being no competition for resources, so there is no change in price level.
\section{Macroeconomic indicators}
\subsection{GDP and GNP}
National income statistics indicate how much is being produced in an economy. Of course, the more the economy produces, the better its performance.

Gross domestic product (GDP) is the monetary value of final goods and services produced within a country over a year.

Gross national product (GNP) is the monetary value of final goods and services produced with the resources of a country over a year. Net national product (NNP) is GNP with depreciation deducted.

Real income (GDP or GNP) is simply income adjusted for inflation.

NNP is the most accurate measure of national income. However, countries typically use GNP as it is difficult to accurately determine the value of depreciation.

National income statistics can be used to compare standard of living of a country at two instants in time, or between countries. When this is done, we typically use real GDP or GNP per capita, which is simply GDP\slash{}GNP divided by the number of residents. An increase in GDP or GNP per capita indicates generally an increase in standard of living of residents.

When comparing across countries, we may use a purchasing power parity (PPP) exchange rate, in order to account for differences in the price levels and prices of common items and necessities. A PPP exchange rate is a rate of exchange that would allow a given amount of money in one country to buy the same amount of goods in another country after exchanging it into the currency of that other country.
\subsubsection{Limitations of GDP and GNP}
While GDP and GNP can give an indication of the standard of living of countries, it is not entirely accurate. We categorise problems into measurement problems and usage problems i.e.\ problems about the accuracy of GDP and GNP, and problems about what they do not even measure.

GDP or GNP per capita may be inaccurate simply due to the information gathered being inaccurate e.g.\ in developing countries with low levels of accuracy, the people may not be able to declare their income accurately in tax forms, understating the actual output produced by the economy.

They also do not take into account non-marketed goods and services, especially goods produced that are not traded, like voluntary work and goods produced for oneself, and goods produced that are exchanged without money i.e.\ barter trade. This makes the GDP and GNP an understatement of the well-being of the society.

If economic activities are illegal e.g.\ because they were done in a black market, they will not be reported either. If the black economy has been growing at a significant rate, it may mean income statistics understate the increase in material well-being of consumers. Also, people may not declare income because they wish to avoid paying tax.

Even if GDP and GNP are accurate, they do not give any indication about the income distribution. Thus an increase in income per capita does not mean everyone in the country is better off, as the income distribution in the country may be very bad. If data like the Gini coefficient is considered along with GDP and GNP, it may give a better account of how an increase in real income is distributed among the population.

GDP and GNP do not consider the type and quality of products and how much they actually increase the well-being of the people. Typically, goods that increase well-being include necessities like housing, food and clothing. During a war, for example, income could increase due to an increase in production of weapons, but no one will say their well-being has increased. Also, the quality of products has generally increased over time especially as technology has improved, but this is not reflected in income either.

Income statistics also do not at all consider the non-material welfare of individuals i.e.\ quality of life. Quality of life considers factors like a person's environment, crime rates of a country, security, life expectancy, healthcare, quality of leisure and even happiness. For example, an increase in real income per capita in a country may overstate the increase in standard of living if the increase in income is because people are working longer, leading to more stress and less quality leisure time.

When using GDP and GNP to compare standard of living across countries, there are more problems that arise.

Even if a country has a higher real income per capita compared to another country, comparing using a common currency, it does not mean the former country's citizens have a better standard of living, because the price levels or cost of living in the first country may simply be higher. This can be accounted for by using a PPP exchange rate.

Also, the reliability of data often differs across countries. Income data from developing countries may be inaccurate due to lack of facilities for and expertise in data collection, which makes it difficult to compare standard of living between developing and developed countries --- the standard of living of a developing country may not be as bad as their income data makes it seem.

The size of black markets and the amount of goods produced that are not exchanged in a legal market also differs. Some countries may have larger black markets due to laws like price ceilings while other countries may have significant amounts of output not reported because of e.g. subsistence farming. Even housework is considered a service that, if done by the houseowner, will not be reported. When comparing the standard of living between developing and developed countries, this poses a problem, as a large part of agricultural production in developing countries is not traded --- subsistence farming --- and so is not accounted for in income; thus income will underestimate the country's output and thus standard of living.

Similar to before, the income distribution between the two countries can also differ, which means income will not be a good indicator of the standard of living. For example, if one country has a relatively fairer income distribution but a lower income per capita, that country will have a higher standard of living overall than a country that has an unfair income distribution but a higher income per capita --- it simply means a few rich people have a lot of income, while the rest of the country is poor, but income per capita alone cannot indicate that.

Also, the types of goods produced across countries differ. Even if one country has a higher income per capita than another, if the former country produces goods like weapons that do not really improve standard of living, while the latter country has a large agricultural sector, then the latter country will have a better standard of living, even though the income per capita says otherwise.

The quality of life between countries also differ. A country may have a high income per capita but the people of the country might also be overworked and stressed, and their quality of life will be much lower than another country that might have a lower income per capita but a happier population. In this case, income per capita will overstate how much higher the standard of living of the former country is compared to the latter country.
\subsection{Other indicators}
The Gini coefficient is a measure of the extent to which the distribution of income among households deviates from a perfectly equal income distribution; zero indicates complete equality and one indicates complete inequality.

The Measurable Economic Welfare (MEW) is an adjusted GDP figure that factors in quality of life. Things that improve quality of life add to the GDP while things that reduce quality of life subtract from GDP.

The Human Development Index (HDI) is a composite indicator measuring many things, including GDP per capita in PPP\$, life expectancy at birth, literacy rate and school enrolment. The HDI ranges from 0 to 1 where 0 is the lowest and 1 is the best possible score. Empirical evidence indicates that countries like Canada have a higher HDI than income per capita indicates, and countries like Singapore have a lower HDI than income per capita indicates. Critics point out that the HDI do not take into account issues like freedom, human rights, protection from violence and discrimination.
\end{document}
