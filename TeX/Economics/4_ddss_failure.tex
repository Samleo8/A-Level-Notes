\documentclass[Economics.tex]{subfiles}
\begin{document}
\chapter{Market Failure}
The microeconomic objectives of the government are to achieve efficiency in resource allocation and promote equity in income and wealth distribution.

Market failure occurs when the price mechanism fails to allocate resources efficiently and equitably. An economy is said to be efficient if it has achieved productive efficiency and allocative efficiency.

At the economic level, productive efficiency occurs when it is impossible to make more of one good without making less of another, that is, the economy is operating on the production possibility curve, while allocative efficiency occurs when it is impossible to make someone better off without making someone worse off. In other words, it is when the right amount of the right goods are produced.

A perfectly competitive market without any externalities would achieve social efficiency.

At the firm level, productive efficiency occurs when a firm produces the maximum output for a given amount of input, or a given output at the least cost. Productive efficient firms produce on their long run average cost curves and are using the least cost combination of resources for that output level.

Allocative efficiency at the market level occurs when it produces an output where the marginal social benefit equals the marginal social cost, or in other words, where societal welfare is maximised.

The marginal social benefit (\mrg{SB}) is the value society places on the last unit of a good that is consumed. Likewise, the marginal social cost (\mrg{SC}) is the value of alternative goods society forgoes by producing the last unit of a good.

Similarly, the marginal private benefit (\mrg{PB}) refers to the benefit that consumers place on the last unit of a good that is consumed, and is represented by demand. The marginal private cost (\mrg{PC}) refers to the costs incurred by producers in producing the last unit of a good, and is represented by supply.

When there are no externalities in production or consumption, \mrg{PC} is equal to \mrg{SC} and \mrg{PB} is equal to \mrg{SB}, so the market equilibrium coincides with the socially optimal outcome.
\section{Externalities}
Private costs refer to the costs incurred by the producer or consumer. The private cost of production is usually the cost of the factors of production, like raw materials or labour. The private cost to consumers is usually the cost of purchasing a good as well as goods that are required to consume that good.

Private benefits refer to the benefit or satisfaction received by producers or consumers. The private benefit of production is the revenue received from the production and sale of a good. The private benefit to consumers is the satisfaction obtained from the consumption of a good.

An externality is a cost or benefit that affects someone not directly involved in the production or consumption of a good for which no compensation is provided.

External costs, or negative externalities, refer to costs incurred due to production or consumption of a good by people other than producers or consumers of that good, for which no compensation is provided.

External benefits, or positive externalities, refer to benefits gained due to production or consumption of a good by people other than producers or consumers of that good, for which no compensation is provided.

Social cost measures the next best alternative use of resources that are available to the whole society. It measures the total cost to society of an economic activity and thus is the sum of private cost and externalities in production.

Social benefit measures the satisfaction that society is able to obtain from being involved in a certain activity. It measures the total benefit to society of an economic activity and thus is the sum of private benefit and externalities in consumption.

On a demand and supply diagram,
\begin{slinenum}
\item the demand curve represents the private benefits of an additional unit of output produced and is thus the \mrg{PB}
\item the supply curve represents the private costs of an additional unit of output produced in the industry --- it is the cost that firms take into consideration and is thus the \mrg{PC}
\item the market outcome is found at the intersection of the demand and supply curve, or the \mrg{PB} and \mrg{PC}, as it is where consumer and producer welfare is maximised
\item the socially optimal outcome is found at the intersection of the \mrg{SC} and \mrg{SB}, as it is where societal welfare is maximised.
\end{slinenum}

If there are negative externalities in the \emph{production\slash{}consumption} of a good, the \emph{\mrg{SC}\slash{}\mrg{PB}} will be higher than the \emph{\mrg{PC}\slash{}\mrg{SB}}. At the market output, the \mrg{SC} exceeds the \mrg{SB}, and the last unit's worth to society is less than what society sacrifices for its \emph{production\slash{}consumption}. There will be an over\emph{production\slash{}consumption} of the good, and the social cost of \emph{producing\slash{}consuming} the excess units of the good will exceed the social benefit of that, leading to a welfare loss. Too much resources are allocated to the production of the good, thus there is allocative inefficiency, and the market fails.

If there are positive externalities in the \emph{production\slash{}{\-}consumption} of a good, the \emph{\mrg{SC}\slash{}{\-}\mrg{PB}} will be lower than the \emph{\mrg{PC}\slash{}{\-}\mrg{SB}}. For all units \emph{produced\slash{}{\-}consumed} before the socially optimal output is reached, the \mrg{SB} is higher than the \mrg{SC}, so all these units should be \emph{produced\slash{}{\-}consumed}, but the market is \emph{producing\slash{}{\-}consuming} less than this amount, so the market is under\emph{producing\slash{}{\-}consuming} the good. The social benefit of \emph{producing\slash{}{\-}consuming} the under\emph{produced\slash{}{\-}consumed} quantity is more than the social cost of doing so, thus this under\emph{production\slash{}{\-}consumption} causes a deadweight loss. Too little resources are allocated to the production of the good; there is allocative inefficiency, and the market fails.
\section{Information failure}
Information failure arises when consumers do not get the right information or lack the relevant information on the benefits or harm that they are likely to receive from the consumption of a product.

\emph{De\slash{}merit} goods are goods that the government believes consumers will consume too \emph{many\slash{}few} units of if provided by the market due to information failure and \emph{negative\slash{}positive} externalities in consumption.

Individuals who make decisions about how much of a \emph{de\slash{}merit} good to consume do not fully appreciate the private \emph{costs\slash{}benefits} that will be \emph{incurred\slash{}received} through consuming the good, sometimes because \emph{these benefits are in the future, uncertain or difficult to estimate accurately\slash{}they are not fully informed about the risks of consuming the demerit good}. This lack of information leads to an \emph{over\slash{}under}estimation of the private benefits of the \emph{de\slash{}merit} good.

\emph{De\slash{}merit} goods also exhibit \emph{negative\slash{}positive} externalities in consumption --- the social benefit of consuming a merit good \emph{is less than\slash{}exceeds} the private benefit of doing so. Consumers do not take into account external \emph{costs\slash{}benefits} and they \emph{over\slash{}under}estimate the private benefit of consuming a merit good, resulting in \emph{excessive\slash{}insufficient} demand registered for the good in the market, so the \mrg{PB} is \emph{higher\slash{}lower} than the \mrg{SB}.

If consumers were fully aware of the private \emph{costs\slash{}benefits} to oneself i.e.\ there is perfect information, the \mrg{PB} would be \emph{lower\slash{}higher} than the first \mrg{PB} but still \emph{higher\slash{}lower} than the \mrg{SB}. Since consumers do not take into account the external \emph{costs\slash{}benefits} of consuming the \emph{de\slash{}merit} good, both demand curves are \emph{higher\slash{}lower} than the \mrg{SB}.

\emph{(For merit good)} There is an underproduction and underconsumption of the merit good. Should the underconsumed quantity be consumed, the social benefit would exceed the social cost of doing so and societal welfare would be greater. Thus this underconsumption has led to societal welfare not being maximised i.e.\ there is a deadweight loss to society. Too little resources are allocated to the production of the good, thus there is allocative inefficiency, and the market fails.

\emph{(For demerit good)} There is an overproduction and overconsumption of the demerit good. The social cost of consuming the excess units of the good will exceed the social benefit of that, leading to a welfare loss. Too much resources are allocated to the production of the good, thus there is allocative inefficiency, and the market fails.
\section{Public goods}
Non-excludability of a good means it is either not economically feasible or not possible to exclude anyone from using a good once it is provided. It is not possible to assign property rights to only those who pay for the good, giving rise to the free rider problem, where it is possible for a person to consume a public good without paying for it. To consumers, the possibility of being a free rider weakens the incentive for consumers to pay for the good i.e.\ consumers will not register their demand in the market. Since there is no expression of demand, it is impossible to charge for a public good, and if left to private enterprise, such goods will not be provided at all. There is a missing market for public goods.

Non-rivalry of a good means that the consumption of the product by one additional person does not diminish another person's ability to consume the good. This means that once the product is produced, the additional cost to allow another person to benefit from consuming the product, or the marginal cost of admitting another user, is zero. Since the socially optimal number of users occurs when \(\mrg{SB} = \mrg{SC}\), the socially ideal price to charge each user is zero, so no public good will be supplied by the market since producers are profit motivated.

A public good is a good that exhibits the characteristics of non-excludability and non-rivalry. These characteristics result in a public good not being produced at all in a free market economy, resulting in market failure as no resources are allocated to these goods, which are usually essential and beneficial to society. There is allocative inefficiency in the market, and the market fails.
\section{Market imperfections}
Markets will operate efficiently if they operate under perfect competition, and there are no externalities. Perfect competition means that all factors of production i.e.\ land, labour, capital and entrepreneurship are perfectly mobile and respond perfectly to demand, enabling producers to respond efficiently to price signals, thereby maximising societal welfare.

Market dominance occurs when firms are able to set price above marginal cost to earn large profits that cannot be competed away by rivals. In this case, the market outcome will be allocative inefficient.

A perfectly competitive market would produce at the point where \(D = S\) i.e.\ \(\mrg{SB} = \mrg{SC}\), and societal welfare is maximised. A monopoly, however, would produce where \(\mrg{C} = \mrg{R}\), and at this output \(P > \mrg{C}\). Society values the last unit of output more than it cost to produce it; societal welfare will increase if more is produced. There is an underproduction of the good; the social benefit of producing the underproduced amount is greater than the social cost of doing so, and so this underproduction causes a deadweight loss. The market is thus allocative inefficient.

Markets with strong barriers to entry are also inequitable. Firms in such markets are able to earn supernormal profits as new firms cannot enter the market to compete away profits due to the barriers to entry. Existing firms can charge higher prices and earn more profits than is economically justified, and this supernormal profit will be distributed as dividends to shareholders, who are typically high-income earners (in order to afford shares in the first place). Consumers, who may be low-income earners, may also be worse off as they have to pay higher prices to get the same good. This worsens the income distribution as the rich get richer and the poor get poorer i.e.\ inequity.

In reality, factors may not be perfectly mobile. There are two main types of factor immobility. Occupational immobility occurs when labour, capital or land cannot move from industry to another, due to the lack of relevant skills, or trade union restrictions (in the case of labour), or simply being unsuitable for the target industry. Geographical immobility occurs when factors of production cannot move from one location or place of employment to another, due to lack of information, resistance to change or difficulty of securing new accommodation (in the case of labour), or high costs of moving or transportation.

Occupational immobility of labour can lead to market failure.

Suppose there are changes in demand patterns. There is a decrease in demand for basic electronics and an increase in demand for pharmaceutical goods. This leads to a decrease in demand for workers in the electronics industry and an increase in demand for pharmaceutical workers. The supply of pharmaceutical workers is wage inelastic, as specialised knowledge is needed for that field. The increase in demand for pharmaceutical labour will lead to a large increase in wages, attracting the now-unemployed electronics workers, who are willing but unable to take up the jobs in the pharmaceutical industry, as they lack relevant skills. These unemployed electronics workers will be structurally unemployed. The higher wages in the pharmaceutical industry will cause firms in that industry to incur higher costs and hire fewer workers than if labour were fully mobile, resulting in productive inefficiency as firms are unable to utilise the least cost method of production. Since firms are unable to acquire more labour easily, they are less able to respond to price signals. There will be an underproduction and underconsumption of pharmaceuticals, as less is produced compared to if labour was fully mobile, so the right amount of the right good is not produced, and there is allocative inefficiency. Since there is both allocative and productive inefficiency, the market fails.
\section{Inequity}
Inequity refers to an allocation of resources that is considered unfair and unjust.

Income is a flow of earnings over a specified period that comes from earned income --- wages --- and unearned income, which is income received from savings and shares.

There is generally always income inequality in the real world, which results in inequity. The demand for skilled labour is generally higher than that of unskilled labour, as the former tends to have higher labour productivity. Conversely, the supply for skilled labour is generally lower than that of unskilled labour, as the former requires specialised technical skills that are not easily acquired. A combination of high demand and low supply for skilled workers leads to high wages and thus high income for skilled workers, while the reverse holds true for unskilled workers. There is thus income inequality between skilled and unskilled workers. Due to occupational immobility, unskilled workers cannot move into the skilled labour market as they simply lack the skills required; they are thus unable to earn the higher wages that skilled labour do. Unequal incomes persist, leading to income inequality. The presence of trade unions can also lead to income inequality, as workers that are members of large trade unions will be able to earn higher wages than non-unionised workers do due to the collective bargaining power possessed by large unions.

Income inequality results in inequity. In a market-based economy, allocation of resources is based on the price mechanism. An increase in demand for a good causes an increase in prices, and producers respond to this by increasing production (and thus supply). The goods produced are distributed among those who are willing and able to pay. Individuals who are unable to pay will not have access to goods and will be rationed out of the market, and so in a market based economy, the types and amount of goods produced is based on the willingness and ability to pay, rather than on needs. If income is unevenly distributed, those with more income will be able to affect what and how much goods are produced, as they can cast more dollar votes for the goods they want. Resources will be allocated more to produce luxury goods for the rich, while the needs of the poor will not be fulfilled. The rich will be able to enjoy more goods and services compared to the poor as they can demand more goods and services, having more dollar votes, so more resources are allocated to production of goods and services for the rich. There is thus unfair distribution of resources as the rich have access to more necessities than the poor do, and thus the market outcome is inequitable. The market fails.
\end{document}
