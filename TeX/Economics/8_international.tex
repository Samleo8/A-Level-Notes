\documentclass[Economics.tex]{subfiles}
\begin{document}
\chapter{International Trade}
International trade refers to the exchange of goods and services between countries.
\section{Theories explaining trade}
\subsection{Theory of comparative advantage}
The theory of comparative advantage states that trade can benefit countries involved if \begin{slinenum}
\item they specialise in producing and trading those goods in which they have a comparative advantage
\item the terms of trade lies between the domestic opportunity cost ratios of the two countries.
\end{slinenum}

A country is said to have comparative advantage in the production of a good when it can produce that good at a lower opportunity cost than another country; it reflects the country's factor endowment.

To illustrate the theory of comparative advantage, we assume that \begin{slinenum}
\item there are two countries A and B that produce rice and cloth
\item resources within the country are fully mobile
\item there are constant costs in both industries
\item there is full employment
\item there is no transport cost
\item there are no barriers to trade
\item each country has a given amount of resources.
\end{slinenum}

Suppose that in the absence of trade, both A and B devote half their resources to each product. The output produced for each product in each country before specialisation is given below.

{\centering\begin{tabular}{ccc}
\toprule
\textbf{Country} & \textbf{Cloth} & \textbf{Rice} \\
\midrule
A & 30 & 15 \\
B & 5 & 10 \\
Total & 35 & 25 \\
\bottomrule
\end{tabular}}

It can be seen that A has comparative advantage in cloth production, as the opportunity cost of producing a unit of cloth in A is \num{0.5} units of rice, while that in B is \num{2} units of rice. On the other hand, B has comparative advantage in rice production, as the opportunity cost of producing a unit of rice in A is \num{2} units of cloth, while that in B is \num{0.5} units of cloth.

These differences in comparative advantage form the basis of specialisation and trade. Consider the case where the countries specialise in the production of the commodity in which they have a comparative advantage in. Suppose that A devotes two-thirds of its resources to cloth production and one-third to rice production while B fully specialises in rice. The production aftter specialisation is given below.

{\centering\begin{tabular}{ccc}
\toprule
\textbf{Country} & \textbf{Cloth} & \textbf{Rice} \\
\midrule
A & 40 & 10 \\
B & 0 & 20 \\
Total & 40 & 30 \\
\bottomrule
\end{tabular}}

In A, the opportunity cost of producing one unit of cloth is half a unit of rice. For A to gain from trade A must receive more than half a unit of rice for each unit of cloth. Similarly, B will not offer more than \num{2} units of rice per unit of cloth. The terms of trade will thus be between the opportunity cost ratios of A and B i.e.\ between half to two units of rice per unit of cloth.

Assume that the countries decide to trade one unit of rice for one unit of cloth, and they trade \num8 units of rice for \num8 units of cloth. The consumption of both countries is shown below.

{\centering\begin{tabular}{ccc}
\toprule
\textbf{Country} & \textbf{Cloth} & \textbf{Rice} \\
\midrule
A & 32 & 18 \\
B & 8 & 12 \\
Total & 40 & 30 \\
\bottomrule
\end{tabular}}

It can be seen that \begin{slinenum}
\item the world output of rice and cloth has increased by \num5 units each
\item both countries have increased their consumption of both rice and cloth, and so the material well-being of the people in A and B has increased.
\end{slinenum} Trade is thus mutually beneficial for both country A and B.
\subsubsection{Limitations}
The theory of comparative advantage makes some assumptions that are unrealistic in real life.

It assumes that there are constant cost conditions i.e.\ as a country produces more for export, the average costs of production stay the same. This is untrue in reality and diseconomies of scale may occur, resulting in lower gains from trade. Countries will still benefit from trade, but they will not specialise to the extent the theory suggests they will.

Also, the theory assumes factor mobility, which is of course untrue in reality. As a country specialises in the production of a good, it may end up having to use labour that is less suitable in terms of skill for the production of the good, putting a constraint on the possible increase in output level with specialisation, and thus the gains from trade.

There are also transport costs involved when trading, especially when trading with countries that are far away. This reduces the comparative advantage of a country as countries that import from that country will consider the cost of transporting goods from that country in addition to the price of the goods.

Finally, in the real world, there also exist trade barriers that reduce the volume of trade.
\subsection{Other explanations for trade}
The theory of comparative advantage uses cost difference as a reason for trade. However, there are other reasons.

Differences in demand can be a reason to trade. Suppose there are two countries A and B that both produce meat and vegetables equally efficiently, but A has more meat-eaters while B has more vegetarians. If A and B do not trade, meat will be relatively more expensive in A while vegetables will be relatively more expensive in B. Trade between the countries will reduce the price of meat in A and price of vegetables in B, and will also increase the incomes of meat farmers in B and of vegetable farmers in A.

Another reason could be seasonal produces, like fruits. Suppose a country A produces strawberries, and has relatively constant demand for strawberries throughout the year. When A's strawberries are off-season, demand is relatively higher than supply, so A imports strawberries from overseas. When A's strawberries are in season, demand is relatively lower than supply, so A exports strawberries.
\section{Benefits of trade}
The main benefit from trade is the gain in welfare. It can be shown that when a country imports a good at a price lower than its domestic market price, the consumer surplus increases while the producer surplus decreases, and the increase in the former is more than the decrease in the latter such that there is a net increase in welfare.

Trade also allows people to have more choices in terms of the branding or variety of products. Overall, the availability of products at cheaper prices and more choices leads to higher material well-being and thus standard of living.

Trade can also allow a country to enjoy economies of scale as it is producing for a larger international market, allowing industries to produce at lower unit costs, which will be passed on to consumers as lower prices.

Competition from imports can also stimulate greater research and force firms to be more efficient, lowering prices and making better quality products.

The disadvantages of trade are largely similar to those of globalisation.
\section{Protectionism}
Protectionism is a deliberate government policy to erect trade barriers in order to shield domestic industries from foreign competition. It aims to switch expenditure, both domestic and foreign, to the output of goods and services of the domestic economy.
\subsection{Tariffs}
A tariff is a tax on imports. When imposed, a tariff will increase the unit cost of supplying an import to domestic consumers, shifting the world supply curve in the domestic economy leftwards, increasing the price of the good overall. It can be seen that when this is done, domestic production increases, but a welfare loss is incurred.

An advantage of tariffs is that they, like all other taxes, provide a source of income for the government, which in the context of free trade can be used to help make the country more competitive in the world economy.

Tariffs, however, depend on the \PE[D] and \PE[S] of domestically produced substitutes. If the demand or supply for domestically produced substitutes is inelastic, then the effectiveness of a tariff is reduced, and a larger tariff is needed to bring about a significant increase in domestic production.
\subsection{Quotas}
A quota is a quantitative restriction on imports. It can limit based on the value or the quantity of the good -- either a maximum value or maximum quantity. When imposed, a quota will reduce the quantity supplied of imports in the domestic market, increasing the domestic price of the good. Domestic production increases, but a welfare loss is incurred.

An advantage of quotas is that foreign produces cannot overcome them by reducing prices.

However, they result in higher prices in the domestic market, resulting in greater profits for the foreign firms at the expense of domestic consumers, ceteris paribus. Also, the government does not get any revenue from imposing a quota as opposed to a tariff. The government can auction off import licenses which will afford them some revenue. Holders of these licenses can buy the good at the world price and resell them in the domestic market at higher prices, earning a profit.
\subsection{Foreign exchange controls}
A foreign exchange control is a limit on the amount a citizen can deal in foreign currrency. This will cause imports to fall to a level below the free market level, and consumers suffer.

This incurs large administrative costs in monitoring the amount citizens deal in foreign currency.
\subsection{Subsidies}
A subsidy is a payment from the government to firms when they produce a good. Subsidies can be used to help domestic firms compete with foreign producers by reducing their unit cost of production, shifting domestic supply to the right, increasing domestic production at the world price.

A subsidy will cause redistribution of income towards producers as their inefficiencies are covered up for through the taxpayers, who fund the subsidy. Producers who might have shutdown or left the industry are encouraged to remain in the industry as their products become competitive in the international market due to the subsidy, which prevents resources from being reallocated to other industries in which the country has a true comparative advantage.

A subsidy may lead to a domestic monopoly, as foreign competitors are excluded, which may lead to consumer exploitation through raising prices and reducing consumer surpluses. Costs may also rise unnecessarily as the monopoly becomes complacent. However, workers in that industry may benefit as production will be higher than without a subsidy and so they may enjoy higher wages.

An export subsidy can also be used as protectionism, where a country's government subsides goods that are produced for export, reducing the world price. Foreign consumers and domestic workers in subsidised industries benefit, but taxpayers who fund the subsidy lose out. Firms may also divert more resources to production of exports, which may lead to supply in the domestic market falling, leading to higher prices and lower welfare for domestic consumers.
\subsection{Reasons for protectionism}
\subsubsection{Infant industry argument}
A government may erect protectionistic measures in order to protect an infant industry while it is growing. An infant industry is a new industry that a government believes has potential comparative advantage but has yet to realise it. By applying protectionism, the domestic output increases and firms in the infant industry get to produce more, exploiting economies of scale that reduce their long run average cost, earning more profits that can be used for the research and development of new, better methods of production, and simply learning by doing. Eventually and in theory, the infant industry will catch up with more established foreign firms when it realises its comparative advantage, and it will be able to produce the good at a lower opportunity cost than another country. This has benefits for the country as well: as the industry grows, ceteris paribus there will be higher employment and lower prices for the goods the industry produces.

Protectionism has been used by countries like South Korea, Japan and Taiwan to develop their infant industries. However, there are still problems with the use of protectionism in tis way. It is difficult to correctly identify which industries actually have potential comparative advantage, and identifying the wrong industry can be costly for the country as taxpayer income and resources will be directed wrongly. In the short run, consumers will be faced with higher prices of goods produced by the industry. The infant industry may also grow to depend on the state for support. 
\subsubsection{Preventing dumping}
Protectionism is also sometimes used to prevent dumping. Dumping is the practice of selling goods in an overseas market at a price below its marginal cost of production or below the price charged in the home market. After dumping, the quantity demanded of domestically produced goods falls and domestic producers see a fall in output, total revenue, and ceteris paribus, total profits. If their profit level drops below normal, they may have to exit the industry. A foreign firm can dump in order to cause domestic firms to exit the market, allowing them to become a monopoly. Governments have to protect their domestic firms from such unfair competition using protectionism.

However, it is not easy to prove dumping, since it is difficult to find out a firm's marginal cost of production from outside the firm. Firms can file a complaint with the World Trade Organisation if they feel foreign firms are dumping their goods in their market, and a tax will be imposed that raises the foreign firm's price until the issue is investigated. However, firms sometimes use this just to buy time to restructure and become more efficient. Also, if a country can permanently import at a lower price from a foreign producer, then the foreign producer is more efficient than the domestic producer. The people in the country benefit from having the good at a lower price and so there is no reason for the country to protect its inefficient industries.
\subsubsection{Protecting employment in a sunset industry}
When a country loses comparative advantage in the production of a good, there will be a fall in demand for labour in the sunset industry, which leads to some workers losing their jobs. Suppose there is also a rising demand for labour in another industry, but workers from the sunset industry have the wrong skills to work in the expanding industry, which would have a wage inelastic labour supply due to the skill requirement. When demand rises, the quantity of labour employed will be less than if supply was wage elastic, and the difference in the quantity employed is equal to the number of workers unemployed in the sunset industry. Thus the number of workers looking for a job is equal to the number of jobs available but the jobs cannot be taken up due to skills mismatch, and so there is structural unemployment.

Protectionism can be used in this case to slow the decline of the sunset industry so that workers have time to acquire new skills to be employed in other industries, averting the problem of large structural unemployment.
\subsubsection{Protecting employment in a recession}
When there is a recession, or a depression in world trade, a country's export revenue will fall, leading to a fall in \AD{} and real output. Firms produce lesser goods and so require fewer workers, so \(\AD_L\) falls and unemployment rises. Households earn a lower level of income and so their purchasing power, material welfare and standard of living fall. The country may thus use protectionism to reduce imports and increase domestic consumption to boost \AD{} to reverse this problem and boost employment.

This use of protectionism, however, is susceptible to the beggar-thy-neighbour effect.
\subsubsection{Attracting foreign direct investment}
A country can erect protectionistic barriers to make it more worthwhile for foreign investors to set up a branch in the domestic market than export directly because of high import tariffs or other measures. Thus protectionism can attract foreign direct investment to a country, boosting output and employment.

This risks retaliation from foreign countries, which will diminish the gains from increased foreign investment. There will also be a loss in consumer welfare due to protectionism allowing less efficient firms in the domestic market to survive.
\subsubsection{Correcting a trade deficit}
If a country is experiencing a trade deficit, it may use protectionism to limit its imports and import expenditure, increasing the trade balance.

However, it again has problems of retaliation, and the fact that there is a trade deficit may be a sign that there is a structural problem in the economy.
\subsubsection{Preventing labour exploitation}
Some countries use protectionism against developing countries with cheap labour on the grounds that the workers in those countries are being exploited, and they have an unfair advantage. However, economically, this does not make sense as if labour elsewhere is cheaper, then that place has a comparative advantage and resources in the home country should be directed to where it actually has a comparative advantage instead of trying to protect labour-intensive industries.
\subsubsection{National security}
A country may erect protectionistic measures to ensure that some important industries, such as the defense and weapons industries, can survive, so that for example, in times of war, when other countries may not be willing to export those goods, the country is still able to have weapons and defenses to protect itself.
\subsection{Reasons against protectionism}
In general, protectionism can be detrimental as it may invite retaliation by trading partners, such as erecting protectionistic measures of their own, leading to an overall reduction in world trade, or devaluing their currency, defeating some forms of protectionism.

Protectionism can also lead to firms or industries becoming overreliant on state support, becoming complacent and dynamically and productively inefficient, causing an overall worsening in societal welfare due to higher prices and lower output.

Finally, protectionism is subject to the world multiplier or beggar-thy-neighbour effect. Since one country's imports is another country's exports, when a country uses protectionistic measures against another country, they reduce the exporting country's export revenue and thus aggregate demand. If that country's national income falls as a result, their citizens will import less, and this may mean that the original country's export falls, and their AD falls as well, which will reduce some of the increase in domestic consumption brought about by protectionism in the first place. If protectionism is being used to boost growth or employment, it may become less effective.
\section{Globalisation}
Globalisation refers to the rising volume of economic activities taking place across countries worldwide, including the exchange of goods, services, capital, labour and technology.
\subsection{Forms of and factors causing globalisation}
\subsubsection{Increased trade in goods and services}
There has been an increased volume of trade in goods and services over the years. Singapore's export volume has increased more than twofold in the past decade, for example, and its services are also increasingly demanded for by foreigners. Changing comparative advantages has also caused Singapore to start importing food and manufactured goods from other countries that have gained comparative advantage, like Malaysia and China. Singapore's import expenditure has also about doubled over the past decade.

This increase has been mainly due to trade liberalisation as countries seek to reap the gains from trade based on the theory of comparative advantage, in order to raise the material standard of living of their people.

There has also been a fall in transport costs due to containerisation, which is the global standardisation of shipping containers enabling goods to be transported in bulk over different modes of transport. Trade between distant countries is now economically viable and cheap.

Rapid development of countries, especially of the emerging economies, has led to increasing income and thus demand for normal goods in those countries, which have increased imports by those countries. The advancement of information and communication technology and the proliferation of the internet have also enabled information and ideas to be transmitted around the world extremely quickly, enlarging global demand by overcoming imperfect information and providing consumers information about goods produced by other countries and producers about suppliers and markets in other countries, leading to rising demand for imports of consumer goods and raw materials.

The World Trade Organisation has played a key role in this, both by encouraging countries to be more open and by helping to resolve trade disputes. Countries that join the WTO also agree to minimise the use of protectionism, increase transparency of trade policies, and provide developing countries special treatment to aid their development.

Countries that join regional or global organisations like ASEAN, the EU, the UN, World Bank and International Monetary Fund are also encouraged to be more open to trade and capital flows.
\subsubsection{Increased trade in financial capital}
There has also been an increase in the sizes of capital and financial accounts across countries worldwide through increased international capital flows across both developing and developed countries. Singapore has been actively attracting FDI in high-end manufacturing, knowledge-intensive industry and exportable services. It has also been increasing investment in other countries like ASEAN, China, and the Middle East. The volume of hot money flowing into Singapore for investment or as deposits in banks and other financial institutions have also increased, due to its status as an international financial centre.

This increase has mainly been due to a deregulation of the financial sector and a reduction of capital controls in many countries worldwide. With deregulation of banking and new financial activities, financial institutions are allowed to diversify activities and introduce new innovative financial products in pursuit of higher revenue and profits, like hedge, pension and real estate investment funds. Since greater financial activity usually leads to economic growth, governments have abolished capital controls i.e.\ the control of financial inflows and outflows from the capital and financial account, resulting in increases in capital trade between countries.

Improvements in technology have also contributed to this, as legal and technical consultation can be easily provided to clients in another country instantaneously through instant messaging or video conferencing. Financial capital can be traded across countries almost instantly with no cost, increasing the supply of financial products as it lowers the cost of supplying them, making financial businesses more profitable. Trade in capital between distant markets is thus more viable.
\subsubsection{Increased movement of labour}
The third form of globalisation is an increased movement of labour developing countries to developed countries, as there is high labour supply but low demand in the former, and low supply but high demand in the latter. This increases the supply of labour in DCs, helping to keep wages in DCs low.

Changes in government policies, for example the relaxation of barriers to labour mobility, and a greater knowledge of opportunities and accessibility, contribute to this increased movement of labour. For example, EU member countries agree to allow any EU citizen to work in any EU member state. Singapore has relaxed its immigration policies to allow a greater inflow of foreign labour to fill up shortages in various industries and keep wages competitive.
\subsection{Benefits of globalisation}
\subsubsection{Boosts economic growth}
The increase in trade in goods and services has led to an increase in net exports for many countries, increasing \AD{} and, assuming the economy is not at full employment, national income, which is actual economic growth. In Singapore, its physical constraints of small land size and a lack of natural resources have become less of a problem by virtue of globalisation, which has helped open up new markets for our exports, increasing foreign demand for Singapore's goods and services. In developing countries like India, remitted income by nationals working overseas enhance households' purchasing power, boosting \AD{} and actual growth. Remittances might also be used to set up small enterprises that contribute to potential growth.

Foreign direct investment and international movement of labour allow the transfer of skills and technology from the developed countries to developing countries and simply between countries in general, boosting the long run \AS{} of many countries.
\subsubsection{Improves balance of payments}
Globalisation helps to improve the trade balance by raising net exports of goods and services as well as net income inflows, thus improving the current account. It also improves net capital inflows due to the increased flow of capital across countries, improving the capital and financial account. The increase in capital flows also strengthens the exchange rate of open economies, reducing imported inflation.
\subsubsection{Keeps wages competitive}
The increased movement of labour across borders means workers will migrate from countries with high labour supply to those with low labour supply, boosting the supply of labour and keeping wages low, which helps to keep those countries competitive.
\subsubsection{Increases firms' scope of profits}
Globalisation has allowed firms to increase their levels of production and reap economies of scale as they export more, due to lower trade barriers, on top of domestic sales. It also allows firms to outsource operations to reduce cost of production. For example, Singapore Airlines outsources its IT processing and customer service to firms in India and the Philippines to take advantage of lower labour costs.
\subsubsection{Improves efficiency}
Developed countries typically have relative shortages of labour and financial capital reflected by higher wage and interest rates respectively. Globalisation allows these shortages to be fulfilled by the movement of labour from countries that are relatively abundant in labour and capital, reflected by lower wage and interest rates, which is in effect a reallocation of resources to produce goods that are more valued, thus making the allocation of resources more efficient. This can also be applied to labour in different industries, when one industry is expanding and requires more labour, and another is contracting and requires less.

Globalisation also increases the level of competition faced by firms as markets are easier to contest with international trade. Monopolies and oligopolies are thus forced to behave more competitively, reducing inefficiencies like X-inefficiency, giving consumers higher welfare.
\subsubsection{Increases variety}
The increased sharing of ideas and trade brings new goods to each country, increasing the variety enjoyed by consumers, thus improving consumer welfare as they have more choices.
\subsubsection{Improves workers' skills}
Workers who work for MNCs that set up foreign branches will be exposed to the good practices of those MNCs, improving their skills.
\subsection{Challenges of globalisation}
\subsubsection{Increased volatility in economic growth}
Globalisation increases the vulnerability of countries to external shocks. When a major economy like the US or China experiences a depression, demand for many other economies' exports falls, thus causing those economies' \AD{} and thus national income to fall, leading to negative actual growth, causing a recession. This also impacts business confidence, also reducing foreign direct investment and thus capital accumulation, affecting potential growth in the long run.
\subsubsection{Increased volatility of price level}
Globalisation has also made open economies, especially those that have little resources of their own, have more volatile general price levels. When there is high inflation in other countries, the price of their exports increases and countries that depend on those exports will see an increase in prices, which is imported inflation. The horizontal segment of \AS{} shifts upwards, increasing general price level.
\subsubsection{Structural unemployment and income inequality}
When a country loses comparative advantage in an industry, demand for goods of that industry from that country will fall -- both domestic and external. Output by firms in that industry falls, and thus those firms require less labour, retrenching some workers. If these workers do not have the right skills to work in another industry, they will become structurally unemployed. The country refocuses its economy towards goods in which the country has a comparative advantage, causing workers and firms in industries that the country does not have a comparative advantage in to lose out.

Many developed countries like Singapore have lost their comparative advantage in manufactured goods due to cheap labour from countries like India and China. Because of this, the demand for less-skilled labour falls, and their wages may fall (although in reality wages may be sticky downwards), increasing income inequality. This is reinforced by workers from India and China emigrating to developed countries, which increases the supply of less-skilled labour, further decreasing their wages and exacerbating income inequality.
\subsubsection{Volatile balance of payments and exchange rates}
Globalisation results in an increased volatility of the balance of payments. In 1997, when financial institutions in Thailand failed due to failed investments in the property market, hot money withdrew from the entire East Asia, leading to large capital outflows that caused severe balance of payments deficits and depreciation of their currencies.

Income earned by foreign workers as well as the profits and earnings of foreign-owned firms also worsen the current account balance when they are remitted back to their home countries, which affects small, open economies like Singapore.
\subsubsection{Exhaustion of resources}
Globalisation imposes environmental costs by causing a rapid exhaustion of resources, like essential minerals, especially when countries owning these resources now not only produce for themselves but also for countries that buy their exports. This will adversely impact productive capacity in the long run.

Developing countries also do not have very many laws against pollution, causing large external costs in the production of some goods, in terms of carbon emissions and the contribution to global warming. This is worsened because developed countries do have laws against pollution, and so MNCs set up factories in places that do not have such laws to take advantage of that fact.
\subsubsection{LDCs lose out}
LDCs generally do not gain as much from globalisation compared to DCs as globalisation is generally used to take advantage of LDCs being LDCs, with FDI used to set up low-skilled jobs that contribute little progress to the country. Many MNCs are also accused of underpaying workers. Globalisation thus allows MNCs to earn more profits at the expense of LDCs.

Some MNCs have grown so large that governments depend on them to create jobs or for financial support. This allows MNCs to lobby LDCs' governments to craft policies to their advantage, such as tax cuts or relaxation of labour market or environmental regulations. MNCs also sometimes consume resources from LDCs to support their parent countries, and high-polluting firms from DCs may relocate to LDCs so that DCs can benefit without suffering from pollution.

Globalisation has also worsened income inequality between DCs and LDCs, because DCs have had their growth boosted far more than LDCs have, for the reasons above. Another reason is that LDCs generally get poor terms of trade due to their reliance on primary industries that see declining product prices.
\end{document}