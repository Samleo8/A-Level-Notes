\documentclass[Economics.tex]{subfiles}
\begin{document}
\chapter{Government Intervention}
\section{Indirect taxes}
An indirect tax is a solution to correct negative externalities in production and in consumption, and demerit goods. Indirect taxes are compulsory payments made to the government for the provision of goods and services. The amount of tax to impose should be equal to the \mrg{EC} at the socially optimal output.

The tax will cause producers to internalise the external costs by making producers pay for the external cost generated. This will increase the unit cost of supplying a good. There is now less potential profit for each unit of output produced. Supply falls, and the market reaches a new equilibrium where the output is reduced, possibly to the socially optimal level. If the output is equal to the socially optimal level, deadweight loss is eliminated and societal welfare is maximised, so the market is now allocative efficient.

If there is an undertax, the output will still be higher than the socially optimal level, but deadweight loss is still reduced, and societal welfare increases. If there is an overtax by an amount smaller than the MEC/IF, output will be lower than the socially optimal level, but deadweight loss is still reduced. If there is an overtax by an amount greater than the MEC/IF, output will be lower than the socially optimal level, but deadweight loss is increased instead.

Indirect taxes have their advantages and disadvantages.

Indirect taxes are market-based and so are easily implemented without excessive government monitoring compared to other measures like legislation. It has greater flexibility and fairness compared to legislation as the amount of taxes can be varied to reflect the size of the external cost.
Indirect taxes bring in revenue for the government that can be allocated for other uses, like compensating third parties affected by the negative externality in production.

Indirect taxes give firms an advantage to develop cleaner technologies, as a cleaner technology reduces the amount of tax a firm has to pay.

However, successful taxation depends on the accuracy of measuring the exact value of the external costs, which is typically difficult as external costs may be difficult to monetise, especially if external costs cross national boundaries (like pollution) or affect third parties that are affected by a multitude of other factors, like farmers. In other words, taxes are susceptible to government information failure.

If the demand for a good is price inelastic, the value of the tax will be high in order to reduce the output to the socially optimal level. This may lead to illegal activities, like tax evasion or a black market, as people make underground transactions to avoid taxes.

Indirect taxes may lead to inequity if applied to necessities like energy, as a blanket tax will affect everyone, whether rich or poor, and so will generally have a greater impact on the poor.
\section{Carbon permits}
Tradable carbon permits are a solution to correct negative externalities in production of energy. They are rights to produce carbon issued by the government to limit the amount of pollution that firms can discharge, as production activities are usually the major source of carbon emissions.

Tradable permits operate on a cap and trade principle, where firms purchase a permit for every tonne of carbon emitted. The government caps the overall emissions at what it considers the socially ideal level of pollution and the market decides on the price to allocate the permits among firms. Firms that can reduce emissions cheaply will have an incentive to reduce emissions and sell excess permits to firms that cannot do so. Firms that cannot reduce emissions cheaply may find it cheaper to purchase permits from other firms, assuming that the penalty of exceeding their allowed amount of pollution is greater than the cost of purchasing permits. The costs of the permits will add on to a firm's \(\mrg{PC}\) and so firms internalise the external costs. The unit cost of production increases, and there is less potential profit for each unit produced, so \(\mrg{PC}\) falls and is closer/equal to the \(\mrg{SC}\), reducing the deadweight loss.

Low-polluting firms may gain a competitive advantage and drive some high-polluting firms out of business, further reducing the external cost by reducing the amount of emissions overall.

Tradable carbon permits have their advantages and disadvantages, mostly similar to those of taxes.
\section{Research and development}
Research and development subsidies are a solution to negative externalities in production.

Governments subsidise firms' investments into developing better methods of production. If research is successful, the external cost decreases or is possibly eliminated, reducing deadweight loss and reducing or solving the market failure.

Research and development has its advantages and disadvantages.

R\&D does not have any effect on firms' profits, so greater investments are encouraged. Furthermore, if R\&D is successful, firms will enjoy a lower unit cost of production, improving the price competitiveness of their goods.

However, R\&D can be costly, as funds and manpower needs to be redirected to perform R\&D. R\&D is also uncertain and may not result in any positive outcome, wasting any resources used.
\section{Legislation}
Legislation is a solution to correct both negative and positive externalities and merit and demerit goods. They are rules and regulations that individuals and firms must comply with, or face penalties.

If there is an overconsumption or overproduction, a legislation can, for example, reduce demand by making it illegal for minors to purchase the good (in the case of alcohol), or reduce supply by setting laws that limit the amount of pollution firms can generate (in the case of energy). If there is an underconsumption or underproduction, a law can, for example, increase demand by making it mandatory for the good to be consumed, like in the case of vaccinations or education. Should individuals or firms violate these laws, they will face punitive measures; the government will have to monitor to ensure that they are obeying the laws.

Legislation has its advantages and disadvantages.

Legislation is generally easier to devise compared to market-based solutions, as it does not require much market data to decide, for example, how much tax to impose.

Laws, however, require monitoring and enforcement for them to be effective, which can be very costly, requiring a lot of manpower which involves a high opportunity cost. The punishments also need to be severe enough for measures to be effective. Laws are also considered a blunt instrument, as they are not sensitive and usually not customised to the individual needs and circumstances of firms and customers. In many cases, the implementation of laws involves bureaucracy, which can dramatically slow down the process.
\section{Indirect subsidies}
Indirect subsidies are a solution to correct positive externalities, merit goods and inequity. Subsidies are compulsory payments \emph{by} the government to firms for the production of goods and services.

A subsidy will lower the unit cost of production of supplying a good to the market, resulting in a higher potential profit per unit. Supply increases, possibly to the \(\mrg{SC}\). Deadweight loss is reduced or eliminated, reducing or correcting the market failure.

Some subsidies can be given out in the form of means-tested benefits, where the subsidies are only given to people that can prove that their income is below a certain level, to correct inequity.

Indirect subsidies have similar advantages and disadvantages to taxes, where applicable.

However, instead of bringing in revenue, they require government spending. This means that funds need to be directed away from other uses, like education or infrastructure development, in order to provide the subsidy; there is opportunity cost in doing so, and if the cost is higher than the benefits of doing so, the government fails.

Subsidies also tend to increase equity instead of bringing about inequity (unlike indirect taxes), as they lower the price of goods, allowing the poor easier access to them; if the good is a necessity, it helps to increase equity.
\section{Direct provision}
Direct provision is a solution to correct positive externalities, merit goods as well as public goods. Direct provision can be seen as a full subsidy, reducing the \(\mrg{PC}\) curve to zero at all prices. Direct provision is simply the government providing a good itself at no cost to consumers.

Direct provision has similar advantages and disadvantages to subsidies. However, it also has the problem of possibly leading to overconsumption (since it is free), and may lead to the problems incurred by increasing direct taxes like personal income tax, if income tax has to be raised to fund direct provision.
\section{Provision of information}
Provision of information or public education is a solution to correct information failure in merit and demerit goods. It is the government educating the public through mass media, carrying out campaigns to show citizens the importance or detriments of the merit or demerit good (respectively).

By educating the public, individuals will be more aware of the benefits or cost to themselves, and the \(\mrg{PB}\) will increase or decrease accordingly, correcting information failure and reducing the deadweight loss, helping to slightly alleviate the extent of market failure.

Public education is effective mainly in addressing information failure.

It is good because it targets the root cause of the problem --- simply that citizens are unaware.

However, it tends to be costly as resources need to be dedicated to designing campaigns and advertisements, which requires government spending, and so it brings about all the possible risks related to government spending and opportunity cost. Coupled with the fact that public education is uncertain --- citizens may simply ignore campaigns --- it may not be a very suitable measure if the extent of market failure is very serious.
\section{Measures against factor immobility}
Education and training is the primary method to correct occupational immobility of labour; it also corrects inequity. It involves governments giving subsidies and tax rebates to firms for them to provide training and upgrading of skills for their workers; personal income tax rebates are also given to workers themselves if they seek training.

Workers acquire relevant or new skills and then are able to move to new jobs should demand patterns change, reducing occupational immobility. Labour supply becomes more wage elastic. Since the problem of labour immobility is reduced or solved, the problem of productive and allocative inefficiency also reduced or solved. Analyse accordingly.

Education and training has its advantages and disadvantages.

It is good as it addresses the root cause of occupational immobility.

However, it is costly as it involves the loss of output and thus revenue of firms, when workers are unable to work, as they have to attend training. The government subsidy involves the same risks associated with government spending. Education and training is also uncertain and workers who are unreceptive during training will simply waste the funds spent on them. Training will take a long time to show its results, as it needs time.

Other methods of reducing occupational immobility of labour include career fairs, where the government provides information on opportunities mainly for unemployed persons to find jobs. This is done through job fairs and job banks. This helps to reduce unemployment, reducing inefficiency.

To reduce geographical immobility, the government can improve and reduce the cost of using transportation systems, so that labour can easily move from one place to another to work. The government can also provide help in relocation and moving households when workers need to move to other places in order to gain employment. These measures are costly.
\section{Progressive direct taxes}
Progressive taxes are taxes whereby the average rate of taxation increases as income increases. A progressive personal income tax is a tax where those with higher income pay a larger proportion of their income to the government, and vice versa.

Such a tax generally is used to redistribute resources from the high-income earners to the lower-income earners: income taxes reduce individuals' disposable income and thus purchasing power, and since those with higher income are taxed more than those with lower income do, the gap in disposable income is generally reduced, allowing for a more equal distribution of income. This reduced income gap reduces the gap in the ability of the rich and the poor to cast dollar votes; the poor will have a more equal chance to consume necessities, reducing inequity.

Progressive income taxes have their benefits and drawbacks.

They may act as a disincentive to work as reducing a worker's disposable income reduces the opportunity cost of leisure. An extra hour of leisure taken involves a smaller sacrifice in income and thus in consumption; since the rich are more likely to put a higher premium on leisure, they may choose to work less if a lot of their income is taxed, reducing labour productivity.

Also, the rich and highly skilled workers tend to be internationally mobile. If a country has high income taxes, they may choose to move to somewhere that has lower income taxes to work, potentially resulting in a `brain drain' where a large number of skilled workers leave the country, reducing labour productivity.
\section{Price controls}
Price controls are a blunt method to correct inequity. A price ceiling is a limit on the highest price that a producer can charge legally, while a price floor is a limit on the lowest price that a producer can charge legally.

A price ceiling on necessities means that the price can be limited to a value that the poor can afford, so that they can purchase necessities, reducing inequity.

Price ceilings generally lead to permanent shortages if below the equilibrium price; price floors generally lead to permanent surpluses if above the equilibrium price. Otherwise, they are completely ineffective.

Price controls have their benefits and drawbacks.

A price ceiling leads to a shortage, where demand cannot be met by supply. Some method will need to be used to ration the limited supply to customers, which could involve biases or high administration costs. An illegal black market may also form, where suppliers sell at higher prices; customers may also be prepared to pay higher prices to ensure they can get the good, defeating the point of the price ceiling.

A price floor leads to a surplus. Some method needs to be used to deal with the excess supply. The government may choose to purchase all the excess supply, but this is costly. It may also lead to a black market where suppliers sell below the price floor, defeating the point of the price floor.
In terms of minimum wage laws, a price floor will lead to unemployment, worsening inequality.
\section{Measures against market dominance}
\subsection{Taxation}
Various types of taxes can be used to control monopolies or other firms with large market power.

A profit tax is a tax on profit, and it does not affect cost, so the firm's equilibrium stays the same.

A lump-sum tax is a fixed sum tax imposed regardless of output. It is a fixed cost and so will not affect the firm's \(\mrg{C}\) (and by extension, output), but it will increase \(\AC\).

A unit tax is simply the same as the indirect tax as above. It will increase both \(\mrg{C}\) and \(\AC\) curves and so it generally decreases output.

Taxes that do not change the market outcome generally do not affect allocative inefficiency, but by decreasing supernormal profits, they improve equity. However, since variable taxes further reduce output, they actually worsen allocative inefficiency. Taxes in general result in lower dynamic efficiency as firms have less incentive to innovate; lower costs that lead to higher profits would simply be taxed away.
\subsection{Legislation}
Legislation can also be used to control monopolies or firms with large market power.

Laws can be set that restrict the market concentration ratio in order to reduce market shares and reduce market power possessed by firms. Laws can also be set to control the price of monopolies e.g.\ forcing all price changes to be approved by the government, to ensure that the consumer surplus is being maintained, or a price ceiling can be set such that quantity at the price ceiling is equal to or closer to the socially optimal output.

Competition policies can be set in place to prevent anticompetitive behaviour, like mergers that result in overly large firms that have too much market power as considered by the government.

Policies to control firms are generally difficult to implement as they can alienate firms; policymakers may also lack the technical knowledge to fairly decide what should be done. Generally, these policies reduce market power, making the firm more allocative efficient and more equitable. The usual problems with laws apply.

Governments can also regulate natural monopolies using \(\AC\)- or \(\mrg{C}\)-pricing, which is simply forcing the firm to set price at \(\AC\) or \(\mrg{C}\) respectively. Generally, this reduces or eliminates allocative inefficiency as the firm's output increases closer to P = \(\mrg{C}\); inequity also decreases as supernormal profits decrease, so consumer surplus increases. However, \(\mrg{C}\)-pricing very often leads to a subnormal profit, so it is usually not viable as no firm can exist in the long run if \(\mrg{C}\)-pricing is forced on such a market.

Two-tier pricing can be used, where consumers pay a standard charge that covers the fixed cost, and then a marginal charge that covers the variable cost depending on how much they use. Each consumer will pay a fixed charge, equal to the difference between average cost and marginal cost at the equilibrium, and a marginal charge equal to the marginal cost at equilibrium for each unit.

For natural monopolies, the state can take over production of the good to protect the interests of consumers. Since states are not profit-motivated, they will produce at levels that are socially optimal; any profits made can be passed onto the country in the form of tax reliefs or lower prices. However, state-owned enterprises may be bureaucratic or unresponsive to consumers' wishes. They will also likely be a tax burden as they do not charge profit-maximising prices and so they may earn subnormal profits, which will be made up using taxes.

The theory of contestable markets suggests, however, that there is no need for governments to intervene, as market power of producers can be constrained by potential competition, forcing firms to keep prices closer to marginal cost and reduce profits because barriers to entry are low, so the monopoly may not be that allocative inefficient or inequitable. Monopolies in contestable markets may benefit consumers more than firms operating in a perfectly competitive market as monopolies can enjoy economies of scale, and the threat of potential competition will ensure that the firms act competitively to keep profits and thus prices low.
\section{Government failure}
Government intervention can sometimes result in the society being worse off. This is known as government failure.

For example, if the government experiences information failure when estimating the extent of the market failure or external cost\slash{}benefit etc., and overcorrects by a large extent, it may worsen the situation. If the form of intervention involves government funding, that funding may be wasted if the government overcorrects.

Governments usually involve a lot of bureaucracy, which leads to time lag, as time is needed to do administrative work, pass any bills in parliament, et cetera. By the time intervention takes effect, circumstances may have changed, requiring different measures.

Government intervention can also result in disincentive effects, e.g.\ reducing the incentive to work, or creating a black market, like detailed above.

Politicians may sometimes do the things that ensure they will win the next election, instead of what would be economically the best course of action, which may lead to a less than optimal use of resources.
\end{document}
