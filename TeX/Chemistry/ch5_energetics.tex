\documentclass[Chemistry.tex]{subfiles}
\begin{document}
\chapter{Chemical Energetics}
\section{Enthalpy changes}
The \sldef{enthalpy} of a substance (\(H\)) is a measure of the energy content of a substance.

The \sldef{enthalpy change} of a reaction (\(\Delta H\)) is the amount of energy absorbed or released in a chemical reaction with respect to the number of moles of reactants or products specified. \begin{equation}\Delta H_{rxn} = \sum H_\text{pdts} - \sum H_\text{rxts}\end{equation}

An \sldef{endothermic reaction} is one in which \(\Delta H > 0\). Energy is absorbed from the surroundings and so the temperature of the surroundings decreases. The products generally have a higher energy content than the reactants and so the products are said to be less stable than the reactants.

A very endothermic reaction is likely to proceed in the reverse direction (as the reverse reaction will be likewise very exothermic), so the products formed may be unstable and have a tendency to decompose or otherwise react to form the reactants of the forward reaction.

An \sldef{exothermic reaction} is one in which \(\Delta H < 0\). Energy is released to the surroundings and so the temperature of the surroundings increases. The products generally have a lower energy content than the reactants and so are more stable than the reactants.

A very exothermic reaction may be explosive as a large amount of heat is evolved; reactions that have only solid or liquid reactants and gaseous products may also be so as there will be a great volume increase.

A reaction being exothermic indicates that the net energy absorbed in breaking bonds in the reactants is less than the net energy released in forming bonds in the products. Exothermic reactions are more thermodynamically feasible than endothermic reactions and so are more likely to occur spontaneously when initiated.

The enthalpy change of a reaction is affected by \begin{slinenum}
\item the number of moles of reactants
\item the states of reactants and products, due to latent heats of fusion and vaporisation
\item the temperature and pressure at which a reaction takes place.
\end{slinenum}
\subsection{Definitions}
The \sldef{standard enthalpy change of reaction} \(\Delta H^\slstdc\) of a reaction is the amount of energy absorbed or released in the chemical reaction when molar quantities stated in the chemical equation react under standard conditions of \SI{298}{\kelvin} and \SI{1}{\atmosphere}.

\sldef{Standard conditions}, referred to by the plimsoll \(\slstdc\) refers to a temperature of \SI{298}{\kelvin} and pressure of \SI{1}{\atmosphere}, with solutions at the concentration of \SI{1}{\mol\per\cubic\deci\metre}.

The \sldef{standard enthalpy change of combustion} \sldHs{c} is the energy released when one mole of a substance is completely burnt in oxygen at \SI{298}{\kelvin} and \SI{1}{\atmosphere}.

The \sldef{standard enthalpy change of neutralisation} \sldHs{n} is the energy released when an acid and a base react to form one mole of water at \SI{298}{\kelvin} and \SI{1}{\atmosphere}.

A neutralisation involving a strong acid and strong base will always have a \sldHs{n} of \SI{-57.3}{\kilo\joule\per\mol}. A neutralisation involving any weak acids or bases will have one that is less exothermic than that, as \begin{slinenum}
\item weak acids or weak bases are only slightly dissociated in aqueous solution
\item so some of the energy evolved from the neutralisation process is used to further dissociate the weak acid or weak base completely
\item this leads to a less exothermic standard enthalpy change of neutralisation.
\end{slinenum}

The \sldef{standard enthalpy change of formation} \sldHs{f} of a substance is the energy change when one mole of the substance is formed from its elements at \SI{298}{\kelvin} and \SI{1}{\atmosphere}. The \sldHs{f} of elements is zero.

The \sldef{standard enthalpy change of atomisation} \sldHs{at} of an element is the energy absorbed when one mole of gaseous atoms is formed from the element at \SI{298}{\kelvin} and \SI{1}{\atmosphere}.

The \sldef{standard enthalpy change of atomisation of a compound} is the energy absorbed when one mole of the gaseous compound is broken down into its constituent gaseous atoms at \SI{298}{\kelvin} and \SI{1}{\atmosphere}.

The \sldef{first ionisation energy} (\slnIE{1}) of an element is the energy absorbed when one mole of gaseous atoms loses one mole of electrons to form one mole of singly charged positive gaseous ions.

The \sldef{second ionisation energy} (\slnIE{2}) of an element is the energy absorbed when one mole of singly charged positive gaseous ions loses one mole of electrons to form one mole of doubly charged positive gaseous ions.

The \sldef{first electron affinity} (\slnEA{1}) of an element is the energy released when one mole of gaseous atoms gains one mole of electrons to form one mole of singly charged negative gaseous ions. The \slnEA{1} is always exothermic as the energy released when the nucleus attracts an electron is greater than the energy absorbed to overcome inter-electron repulsion.

The \sldef{second electron affinity} (\slnEA{2}) of an element is the energy absorbed when one mole of singly charged negative gaseous ions gain one mole of electrons to form one mole of doubly charged negative gaseous ions. The \slnEA{2} is always endothermic as an electron is added to a negative ion; energy is needed to overcome the repulsion between the two negatively charged species.

The \sldef{lattice energy} \sldHs{latt} of an ionic compound is the energy released when one mole of the ionic compound is formed from its constituent gaseous ions at \SI{298}{\kelvin} and \SI{1}{\atmosphere}.

The \sldef{bond energy} \slBE{} of an \slch{X-Y} bond is the average energy absorbed when one mole of \slch{X-Y} bonds in the gaseous state is broken in the gaseous state to form \slch{X} and \slch{Y} gaseous atoms at \SI{298}{\kelvin} and \SI{1}{\atmosphere}. The \sldHs{at} of a non-metal element \slch{X} is equal to half the bond energy of \slch{X-X}.

The \sldef{standard enthalpy change of hydration} \sldHs{hyd} of an ion is the energy released when one mole of the gaseous ion is hydrated at 298 K and 1 atm. \sldHs{hyd} is generally directly proportional to the charge density \(q/r\) of the ion.

The \sldef{standard enthalpy change of solution} \sldHs{sol} of a substance is the energy change when one mole of the substance is completely dissolved in a solvent to form an infinitely dilute solution at \SI{298}{\kelvin} and \SI{1}{\atmosphere}. The \sldHs{sol} indicates the solubility of a substance, as it indicates the enthalpy change of the dissolution process. A more exothermic \sldHs{sol} means the dissolution process of a substance is more exothermic, and thus that the substance is more soluble, and vice versa; a very positive \sldHs{sol} means the substance is insoluble.
\subsection{Calculations}
Enthalpy changes can be calculated from experimental results using some equations.

The energy or heat \(Q\) absorbed or released by a substance to create a temperature change of \(\Delta T\) is given by \begin{equation}Q = mc\Delta T\end{equation} where \(m\) is the mass of the substance and \(c\) is the specific heat capacity of the substance.

When calculating enthalpy changes from experimental results, we assume that \begin{slinenum}
\item there is negligible heat loss to the surroundings, as insulation is used, so any energy released or absorbed in the reactions is transformed into or from heat energy, changing the solution's temperature
\item the density of the solution is \SI{1.00}{\gram\per\centi\metre\cubed}, which is approximately that of water
\item the specific heat capacity of the solution is \SI{4.18}{\kilo\joule\per\kilogram\per\kelvin}, approximately that of water.
\end{slinenum}

The enthalpy change of a reaction can be calculated from other enthalpy changes, where appropriate, using certain formulae.

If the \sldHs{f} of all reactants and products are known, \begin{equation}\sldHs{rxn} = \sum\sldHs{f}_\text{pdts} - \sum\sldHs{f}_\text{rxts}\end{equation}

If the \slBE of all bonds broken and formed are known, \begin{equation}\sldHs{rxn} = \sum\slBE_\text{rxts} - \sum\slBE_\text{pdts}\end{equation} Calculating \sldHs{rxn} using this method may produce a value that differs from the actual value as this method produces an approximation: bond energy values are only averages.

The algebraic method involves the manipulation of given equations to form equations that sum up to the equation of the desired reaction.

Hess' law involves using an alternative pathway of multiple reactions to form the same products.
\section{Hess' Law and Born-Haber cycles}
\sldef{Hess' law} states that the enthalpy change for a chemical reaction is the same regardless of the route by which the chemical change occurs, provided that the initial states of reactants and final states of the products are the same.

An energy level diagram is similar to an energy cycle, except that an energy level diagram shows the relative energies of reacting substances.

The \sldef{Born-Haber cycle} is a special energy cycle used to calculate lattice energies. It equates the overall enthalpy change of the atomisation of the elements in the ionic compound, the ionisation energy and electron affinity of the metal and non-metal respectively, and the lattice energy, to the enthalpy change of formation of the compound. In other words, 
\begin{equation}\sldHs{at} + \slIE + \slEA + \sldHs{latt} = \sldHs{f}\end{equation}

The lattice energies determined using the Born-Haber cycle are experimentally true i.e. they are accurate to reality. Lattice energies can also be theoretically determined based on the geometry of the crystal lattice, assuming a pure ionic compound is formed: ions are taken as point charges that exert electrostatic forces on their neighbours in the crystal lattice.

Lattice energies determined using theoretical models are usually in good agreement with experimental values for predominantly ionic compounds. However, for compounds with significant partial covalent character, theoretical values will tend to differ non-negligibly from the experimental values.

The standard enthalpy change of solution of an ionic compound can be determined using an energy cycle, equating the standard enthalpy change of solution to the standard enthalpy change of hydration of both ions in the substance minus the lattice energy of the ionic compound, i.e. \begin{equation}\sldHs{sol} = \sldHs{hyd} - \sldHs{latt}\end{equation}
\subsection{Stability of compounds}
A substance that has a lower enthalpy content than another substance i.e. a substance whose formation from the latter substance is exothermic is said to be energetically stable relative to the other substance.

A substance that is formed through a reaction with a high activation energy is said to be kinetically stable.

A substance can be energetically unstable but kinetically stable. This means that the substance, energetically speaking, should decompose, but the decomposition is so slow that it is negligible.
\section{Thermodynamics}
\subsection{Entropy}
\sldef{Entropy} is a measure of the disorder in a system.

An increase in temperature means that particles have a greater range of kinetic energies and speeds and move more randomly, so there is more disorder and more entropy.

An increase in the number of moles of liquid due to a solid melting, or an increase in that of gas due to a liquid boiling, means that there are more ways in which particles can arrange themselves i.e. particles are more randomly arranged, so there is more disorder, and thus more entropy.

When substances are mixed, there is always an increase in disorder and entropy, as there are more ways for particles to arrange themselves.
\subsection{Gibbs free energy}
A \sldef{spontaneous process} is one that can occur in a definite direction without being driven by some external force. Whether a reaction is spontaneous depends on the change in \sldef{Gibbs free energy} \(\Delta G\) given by \begin{equation}\Delta G = \Delta H - T\Delta S\end{equation} When \(\Delta G < 0\), the reaction is spontaneous and feasible; when \(\Delta G = 0\), the reaction is at equilibrium (and is feasible); and when \(\Delta G > 0\), the reaction is not spontaneous and thus unfeasible.

Since \(\Delta G\) depends on temperature, the spontaneity of a reaction also depends on temperature. When
\begin{itemize}
\item \(\Delta H > 0\) and \(\Delta S > 0\), \(T\Delta S > \Delta H \Rightarrow \Delta G < 0\) at high \(T \Rightarrow\) reaction is spontaneous and feasible at high \(T\), v.v.
\item \(\Delta H < 0\) and \(\Delta S < 0\), \(T\Delta S > \Delta H \Rightarrow \Delta G < 0\) at low \(T \Rightarrow\) reaction is spontaneous and feasible at low \(T\), v.v.
\item \(\Delta H > 0\) and \(\Delta S < 0\), \(T\Delta S < \Delta H\Rightarrow \Delta G > 0 \mathrel\forall T \Rightarrow\) reaction is not spontaneous nor feasible \(\mathrel\forall T\).
\item \(\Delta H < 0\) and \(\Delta S > 0\), \(T\Delta S > \Delta H \Rightarrow \Delta G < 0 \mathrel\forall T \Rightarrow\) reaction is spontaneous and feasible \(\mathrel\forall T\).
\end{itemize}

Some reactions may not take place even if they are thermodynamically feasible i.e. \(\Delta G < 0\) as their activation energy may be too high for the reaction to occur at an appreciable rate.
\end{document}