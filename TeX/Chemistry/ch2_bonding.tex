\documentclass[Chemistry.tex]{subfiles}
\begin{document}
\chapter{Chemical Bonding}
\sldef{Chemical bonds} are electrostatic forces of attraction that hold two or more atoms, ions, molecules, or any combinations of these together.
\section{Ionic (electrovalent) bonding}
\sldef{Ionic bonds} are strong electrostatic forces of attraction between oppositely charged ions in a giant ionic lattice structure. They are formed in ionic compounds when metal atoms lose electrons to form cations and non-metal atoms gain electrons to form anions, to achieve a stable noble gas configuration.

Generally, a large amount of energy is required to break ionic bonds. The strength of an ionic bond is directly proportional to the magnitude of the lattice energy of the bond, which is directly proportional to the product of the charges of the ions involved but inversely proportional to the distance between the two ions i.e. the sum of the ionic radii of the two ions. In other words, \[\text{strength of ionic bond} \propto \left|\Delta H_\text{latt}^\slstdc\right|\propto\frac{q_+q_-}{r_++r_-}\] The charge is generally the dominating factor that determines lattice energy and, by extension, strength.

When comparing bond strength, compare
\begin{slinenum}
\item ionic charges
\item radii.
\end{slinenum}
\section{Covalent bonding}
\sldef{Covalent bonds} are strong electrostatic forces of attraction between the nuclei of the atoms and their shared pair of electrons in a simple or giant molecular structure. Covalent bonds are formed when non-metal atoms share valence electrons, forming molecules, to achieve a stable noble gas configuration.

A \sldef{dative bond} or coordinate bond is a covalent bond where the shared pair of electrons is provided by only one of the bonded atoms. The donor must have a lone pair of electrons and the acceptor must have vacant and energetically accessible orbitals to accept the lone pair of electrons.

Elements in period 3 and below may contain more than 8 electrons in their valence shell as they have vacant and energetically accessible \(d\) orbitals to expand the octet structure.

Aside from the above reason, some molecules cannot exist or are unstable due to steric repulsion, or overcrowding of the space around a central atom leading to repulsion between electron clouds of atoms.
\subsection{Bond strength}
Covalent bonding involves the effective overlap of valence orbitals of the two atoms involved. A bond involving the head-on overlap of \(s\) or \(p\) orbitals is a \sldef{\mupsigma{} bond}, while a bond involving the side-on or parallel overlap of \(p\) orbitals is a \sldef{\muppi{} bond}. Generally, all single bonds are \mupsigma{} bonds, while double and triple bonds contain 1 \mupsigma{} bond and 1 or 2 \muppi{} bonds respectively.

Covalent bonds generally require a large amount of energy to break. The strength of a covalent bond is directly proportional to the extent of overlap; thus \mupsigma bonds are generally stronger than \muppi{} bonds.

The bond energy, bond length, and the radii of the atoms involved are indicators of bond strength. The greater the bond energy or the shorter the length or radii, the stronger the bond, and vice versa. \sldef{Bond energy} is the energy required to break 1 mole of a specific covalent bond in the gaseous state to form gaseous atoms under standard conditions. \sldef{Bond length} is the distance between the centres of two bonded atoms in a covalent bond. Generally, a triple bond is stronger than a double bond, which is stronger than a single bond.

\sldef{Bond polarity} is a measure of how equally electrons are shared between 2 atoms in a chemical bond. It generally depends on the difference in electronegativity between the atoms. \sldef{Electronegativity}, or \(\chi\), is the tendency of an atom to attract electrons towards itself.

Atoms that have the same \(\chi\) (usually, only identical atoms have the same \(\chi\)) will form a covalent bond where the bonding electrons are equally shared, forming a \sldef{nonpolar covalent bond}. Otherwise, the bonding electrons are not equally shared, and a \sldef{polar covalent bond} is formed.

A polar covalent bond will create a \sldef{dipole moment}, which is a measure of the separation between the positive and negative charges in, and the extent of polarisation or distortion of a covalent bond. Dipole moments are vector quantities that point from the more electropositive atom to the more electronegative atom and have a magnitude that increases with the difference in \(\chi\) between the atoms.

The \sldef{net dipole moment} is the sum of all the dipole moments in a molecule. If there is a net dipole moment i.e. the dipole moments do not cancel off, there will be permanent partial separation of charges; the more electronegative atom has a partial negative charge, and vice versa. The molecule is polar.

When deducing polarity, state
\begin{slinenum}\item the shape of the molecule
\item whether the dipole moments cancel off
\item whether there is a net dipole moment
\item whether the molecule is polar.
\end{slinenum}
\subsection{Ionic bonds with partial covalent character}
Some ionic bonds have a partial covalent character due to distortion or polarisation of the anion's electron cloud by the cation in an ionic compound. The extent of polarisation increases with the cation's charge density and the size and charge of the anion.

When comparing covalent character,
\begin{slinenum}
\item compare the ionic charge and radius and charge density of the cation
\item the ionic charge and size of the anion, where applicable.
\end{slinenum}

Some metal and non-metals, like aluminium chloride, form a covalent bond instead of an ionic bond, due to the cation having a charge density high enough and the anion being large enough that the anion's electron cloud is polarised to such an extent that the bond becomes covalent.
\subsection{Shapes of simple molecules}
The \sldef{Valence Shell Electron Pair Repulsion} (VSEPR) model predicts the shapes of molecules. It is based on the principles that \begin{slinenum}
\item electron pairs around the central atom arrange themselves as far apart as possible to minimise inter-electron repulsion
\item the repulsion between lone pairs is greater than the repulsion between a lone pair and a bond pair, which is greater than the repulsion between bond pairs.\end{slinenum}

To explain the shape of a molecule, \begin{enumerate}
\item state the number of bond pairs (b.p.) and lone pairs (l.p.) around the central atom.
\item state that to minimise repulsion and maximise stability, the electron pairs are directed to\begin{itemize}
\item 2 e.p.: opposite sides of each other
\item 3 e.p.: the corners of an equilateral triangle
\item 4 e.p.: the corners of a regular tetrahedron
\item 5 e.p.: the corners of a trigonal bipyramid
\item 6 e.p.: the corners of an octahedron\end{itemize}
\item state, if there are any lone pairs,\begin{itemize}
\item 1 l.p.: l.p.--b.p. repulsion \(>\) b.p.--b.p. repulsion
\item 2 l.p.: l.p.--l.p. repulsion \(>\) l.p.--b.p. repulsion \(>\) b.p.--b.p. repulsion\end{itemize}
\item state the bond angle and actual shape.\end{enumerate}

If the central atom is more electronegative than the surrounding atoms, bond pairs are drawn closer to the more electronegative central atom. The bond pairs will experience greater repulsion between one another and so the bond angle will be greater, and vice versa.
\section{Intermolecular forces}
\sldef{Van der Waals' interactions} are a class of intermolecular forces of attraction involving attraction between dipoles. There are three types of van der Waals' interactions, namely the \begin{slinenum}
\item induced dipole-induced dipole attraction i.e. London dispersion force
\item induced dipole-permanent dipole attraction i.e. Debye force
\item permanent dipole-permanent dipole attraction i.e. Keesom force.\end{slinenum}

\sldef{London dispersion forces} occur predominantly between nonpolar molecules. They form when electron clouds are temporarily distorted, forming dipoles that induce dipoles in neighbouring molecules, which attract.

\sldef{Keesom forces} occur only between polar molecules. They are simply the electrostatic forces of attraction between the permanent dipoles on polar molecules, and are relatively stronger than London forces.

The strength of van der Waals' forces depends on the size of the electron clouds of the molecules or atoms involved as well as the shape of the molecule. The greater the size of the electron cloud, the greater the extent of distortion of the electron cloud and so the greater the extent of van der Waals' forces of attraction. More energy is required to overcome more extensive van der Waals' interactions.

Straight chain isomers or longer molecules tend to have stronger van der Waals' interactions than branched chain molecules, as the surface area of contact is greater, leading to greater extent of distortion of the electron cloud, causing a greater extent of van der Waals' attraction.

\sldef{Hydrogen bonds} are bonds formed between a hydrogen atom bonded to \slch{N}, \slch{O} or \slch{F} and an \slch{N}, \slch{O} or \slch{F} atom with at least one lone pair. For molecules of similar size, hydrogen bonds are usually significantly stronger than van der Waals' interactions.

The average number of hydrogen bonds formed per molecule depends on the number of lone pairs available on the acceptor (the \slch{N}, \slch{O} or \slch{F} with at least one lone pair), and the number of H atoms attached to \slch{N}, \slch{O} or \slch{F}, whichever is lower.

Some molecules, like ammonia, alcohols, and carboxylic acids, dissolve in water by forming intermolecular hydrogen bonds with water molecules.

Some molecules have rather high melting and boiling points due to the presence of strong intermolecular hydrogen bonds, which require a greater amount of energy to overcome.

Some molecules, like ethanoic acid in an organic solvent (not water, as ethanoic acid will form hydrogen bonds with water), will dimerise due to intermolecular hydrogen bonds forming between pairs.

Some molecules have intramolecular hydrogen bonding, as the donor and acceptor groups are positioned in such a way (close to each other) that hydrogen bonds can be formed within a molecule.

Molecules with intramolecular hydrogen bonding will have lower melting and boiling points than isomers with intermolecular hydrogen bonding, as \begin{slinenum}
\item the formation of intermolecular hydrogen bonds becomes less feasible
\item so phase changes will involve overcoming only the less extensive intermolecular hydrogen bonds
\item this will require a lower amount of energy.\end{slinenum}

Molecules with intramolecular hydrogen bonding will also be less soluble in water than isomers with intermolecular hydrogen bonding, as \begin{slinenum}
\item the formation of hydrogen bonds with water becomes less feasible
\item so there is less extensive hydrogen bonding with water
\item this makes the molecule less soluble in water.\end{slinenum}
\section{Metallic bonding}
Metallic bonds are formed in metals, when metal atoms lose valence electrons to form cations, and the valence electrons become delocalised, free to move within the metallic lattice. The \sldef{metallic bond} is the strong electrostatic force of attraction between the cations and sea of delocalised electrons in a giant metallic structure.

Metallic bonds are generally strong; large amounts of energy are required to break them. The strength of a metallic bond is directly proportional to the number of valence electrons contributed per atom, and the charge density of the metal cation.

When comparing metallic bond strength, compare \begin{slinenum}\item the number of valence electrons contributed per atom \item the charge density of the metal cation.\end{slinenum} \section{Bonding and physical properties}
\subsection{Ionic compounds}
The \sldef{giant ionic lattice structure} consists of oppositely charged ions held together by strong electrostatic forces of attraction (ionic bonds).

The coordination number of an ionic lattice is the number of nearest-neighbour ions to a central ion.

Ionic compounds \begin{slinenum}
\item generally have high melting and boiling points, as a large amount of energy is required to overcome the strong electrostatic attraction between oppositely charged ions
\item cannot conduct electricity in the solid state as ions can only vibrate about their fixed positions and there are no free mobile ions or electrons to conduct electricity
\item can conduct electricity in the molten or aqueous state, as there are free mobile ions to do so
\item are soluble in water as ion-dipole interactions form, which releases energy, causing the detachment of ions from the crystal lattice
\item are insoluble in nonpolar solvents as no ion-dipole interactions can be formed to release energy to break down the crystal lattice
\item are hard as oppositely charged ions are held together by strong electrostatic forces of attraction
\item are brittle as stress applied on an ionic lattice causes layers of ions to slide such that ions of similar charges come together; the resultant repulsion shatters the ionic lattice structure.\end{slinenum}

Ionic compounds can be used as refractories, which can withstand high temperatures due to their high melting points and relative inertness.
\subsection{Giant molecules}
Diamond is an example of a \sldef{giant molecule}, which has strong and extensive covalent bonding between carbon atoms, resulting in a giant three-dimensional tetrahedral molecular structure.

Diamond cannot conduct electricity, as there are no delocalised electrons or free mobile ions to do so. Diamond's strong and extensive covalent bonding between carbon atoms in a three-dimensional molecular structure causes it to \begin{slinenum}
\item have a very high melting point, as a very large amount of energy is required to overcome the bonding
\item be hard
\item be insoluble in both polar and nonpolar solvents as no solute-solvent forces are strong enough to break the bonding.\end{slinenum}

Diamond can be used as abrasives in cutting and boring rocks due to its high melting point and hardness.

Graphite is an example of a molecule with a \sldef{giant layered molecular structure}. In graphite, carbon atoms are arranged in hexagonal flat parallel layers. Within layers, carbon atoms are covalently bonded to 3 other carbon atoms. Adjacent layers are held together by weak van der Waals' forces of attraction.

Graphite \begin{slinenum}
\item has a high melting point, as a large amount of energy is required to overcome the strong and extensive covalent bonding between the atoms in the giant layered structure
\item is a good conductor of electricity parallel to the layers as only three out of four valence electrons of carbon are used for bonding, and the fourth electron is delocalised over the whole layer and are free to move parallel layers to conduct electricity parallel to the layers
\item is a non-conductor of electricity perpendicular to the layers, as electrons cannot flow between layers
\item is soft and slippery as adjacent layers are held by weak van der Waals' forces, so layers can easily slide over each other when a force is applied
\item is insoluble in both polar and nonpolar solvents as no solute-solvent forces are sufficiently strong to overcome the strong and extensive covalent bonding between atoms in the giant layered structure.\end{slinenum}

Graphite can be used as lubricants in e.g. hot machines to reduce friction.
\subsection{Simple molecules}
Iodine is an example of a \sldef{simple molecule}. Atoms of iodine within molecules are bonded by strong covalent bonds, but separate molecules are held together only by weak van der Waals' forces of attraction. In the solid state, iodine molecules are arranged in a regular lattice structure.

Iodine \begin{slinenum}
\item has low melting and boiling points, as a small amount of energy is required to overcome the weak intermolecular van der Waals' forces of attraction
\item does not conduct electricity, as there are no free mobile ions or delocalised electrons to do so
\item is insoluble in polar solvents as the strong intermolecular hydrogen bonds between water molecules are not compatible with the weak van der Waals' forces of attraction between iodine molecules
\item is soluble in nonpolar solvents like benzene as the weak van der Waals' forces between iodine molecules are compatible with the weak van der Waals' forces between benzene molecules.
\end{slinenum}

Ice is another example of a simple molecule. Atoms of hydrogen and oxygen within molecules are bonded by strong covalent bonds, but separate molecules are held together by relatively strong hydrogen bonds.

The presence of two hydrogen atoms and two lone pairs in each water molecule creates a three dimensional tetrahedral structure, making ice not closely packed and forming an open-cage like structure.

Water \begin{slinenum}
\item has a high boiling point, as a large amount of energy is required to overcome the strong hydrogen bonds between water molecules. This is important as otherwise, water would be a gas at r.t.p., and water bodies would not exist, and there would be no rain
\item has a higher density than ice, as when ice melts the tetrahedral structure is partially broken down, causing water molecules to become closer and so there are more water molecules per unit volume, leading to a higher density, which is important as it allows marine life to survive during winter: ice only forms on the top of water bodies and the water below is insulated by the ice layer.
\end{slinenum}
\subsection{Metals}
The \sldef{giant metallic lattice structure} consists of cations in a sea of delocalised electrons held together by strong electrostatic forces of attraction.

Metals \begin{slinenum}
\item have high melting and boiling points, as a large amount of energy is required to overcome the strong electrostatic forces of attraction between the cations and the sea of delocalised electrons
\item are good conductors of electricity as there are delocalised electrons present, which are free to move to conduct electricity
\item are malleable and ductile as when a force is applied, layers of ions can easily slide over each other without the metallic bond being broken; the metallic bond is easily reformed and the crystal lattice is restored.\end{slinenum}

Metals are usually alloyed to make them harder and less malleable, due to different sizes of cations inhibiting sliding. Examples of alloys include bronze, which is made from copper and zinc.
\subsection{Finite resources}
There is a limited amount of natural sources of raw materials like metals.

To overcome shortages of raw materials, humanity has come up with a few solutions. We can
\begin{slinenum}
\item find new sources of materials, through research and discovery
\item develop new and more efficient methods of extracting and refining raw materials
\item recycling used products, by removing and reusing tin from scrap tin cans, copper from scrap electrical wires and pipes, steel from various materials and used cars, and aluminium from used aluminium cans, which is more economical as the extraction of aluminium from bauxite (aluminium ore) requires a lot of energy and is expensive, making recycling more worthwhile.
\end{slinenum}
\end{document}