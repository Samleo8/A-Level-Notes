\documentclass[Chemistry.tex]{subfiles}
\begin{document}
\chapter{Equilibria}
An \sldef{irreversible reaction} is a chemical reaction that proceeds to completion. A \sldef{reversible reaction} is a reaction in which the forward and backward reactions both occur.

Reversible reactions never reach equilibrium, but instead they reach dynamic equilibrium, which is when the rate of the forward reaction equals the rate of the backward reaction. At dynamic equilibrium, substances are still reacting together although the concentration of products and reactants remain constant.
\section{Equilibrium constants}
An \sldef{equilibrium constant} \(K_c\) is the value of the reaction quotient when the reaction has reached equilibrium. The \sldef{reaction quotient} is the product of the concentrations of the products raised to the power of their stoichiometric coefficients, over a similar product of the concentrations of the reactants.

When writing expressions for the equilibrium constant, solid substances, as well as any water acting as a solvent (i.e. there are substances in aqueous state), are omitted.

The equilibrium constant can also be written in terms of the partial pressures of reactants, in which case it is denoted \(K_p\).

Equilibrium constants are affected by temperature. If the reaction is exothermic, then it is varies inversely with temperature; if the reaction is endothermic, then it varies directly with temperature.

The \sldef{degree of dissociation} of a substance is the quotient of the number of moles of that substance that has dissociated over the initial number of moles of that substance.
\section{Factors affecting chemical equilibria}
\sldef{Le Chatelier's principle} states that when a system in equilibrium is subjected to a change or stress that disturbs the equilibrium, the system will react in a way so as to counteract the effect of the change or stress.

If the concentration of a product or reactant is changed, by Le Chatelier's principle, \begin{slinenum}
\item the equilibrium position will shift left or right to increase or decrease the concentration of that substance
\item the concentration of that substance increases or decreases towards the original value until a new equilibrium is attained
\item the new equilibrium mixture will contain more products\slash reactants and less reactants\slash products.
\end{slinenum}

If the pressure is changed, the system will try to change the pressure towards the original value. By Le Chatelier's principle, \begin{slinenum}
\item the equilibrium position shifts left or right in the direction towards a \emph{reduction or increase} in the total number of moles of gas
\item \emph{increasing or decreasing} pressure
\item the new equilibrium mixture will contain more products\slash reactants and less reactants\slash products.
\end{slinenum} If there are the same total number of moles of gas for the reactants and products, then pressure will not affect the equilibrium composition or position.

If the temperature is changed, the system will try to remove or produce heat to shift temperature towards the original value. By Le Chatelier's principle, \begin{slinenum}
\item the equilibrium position shifts left or right towards the exothermic or endothermic reaction to release or absorb heat
\item the new equilibrium mixture will contain more products\slash reactants and less reactants\slash products.
\end{slinenum} The equilibrium constant varies with temperature.

\sldef{Catalysts} increase the rate of forward and backward reaction to the same extent so the position of the equilibrium remains unchanged, but as the rate of reaction increases, equilibrium is reached more quickly.

Le Chatelier's principle can be applied to the Haber process. The \sldef{Haber process} is a process for the manufacture of ammonia from nitrogen and hydrogen, which is an exothermic process.

Usually, the Haber process takes place under a high pressure of \SI{250}{\atmosphere} and moderate temperature of \SI{450}{\celsius}, in the presence of finely divided iron catalyst and with nitrogen and hydrogen in the ratio \(1 : 3\).

By Le Chatelier's principle, \begin{slinenum}
\item an increase in pressure will cause the equilibrium position to shift right towards a reduction in the number of moles to decrease pressure, increasing yield of ammonia
\item an increase in temperature will cause the equilibrium position to shift right towards the exothermic reaction to release heat, increasing yield of ammonia.
\end{slinenum}

However, too high a pressure would be too expensive to maintain, as thicker pipes need to be built to withstand higher pressure. Thus, a pressure of \SI{250}{\atmosphere} is used. Too low a temperature would decrease the rate of reaction until it is too low, so a moderate temperature of \SI{450}{\celsius} is used. There is a compromise between the conflicting demands of high yield and high rate of reaction.

The iron catalyst is used to increase the rate of reaction; it does not affect the yield of ammonia.
\section{Br\o{}nsted-Lowry Theory of Acids and Bases}
An \sldef{acid} is a substance that can donate a proton to another substance in solution. Conversely, a \sldef{base} is a substance that can accept a proton from another substance in solution. A \sldef{Br\o{}nsted-Lowry reaction} involves the transfer of a proton from the acid, which is a proton donor, to the base, which is a proton acceptor.

The \sldef{conjugate base} of a substance is the result of removing a proton from the substance; the \sldef{conjugate acid} of a substance is thus the result of adding a proton to the substance.

A \sldef{strong acid} or \sldef{base} is one that dissociates fully in solution to give protons or hydroxide ions respectively.

A \sldef{weak acid} or \sldef{base} is one that dissociates partially in solution to give protons or hydroxide ions respectively.
\section{Acid and base equilibria}
Pure water can \sldef{self-ionise}, where one water molecule donates a proton to another water molecule. There is thus a small degree of dissociation of pure water. This reaction is endothermic.

The ionic product of water at equilibrium is known as the water dissociation constant \(\Kw\). At \SI{25}{\celsius}, \(\Kw = \SI{10e-14}{\mol\squared\per\deci\metre\tothe{6}}\); \(\Kw\) increases with temperature.

The power of hydrogen \pH{} of a solution is the negative base-10 logarithm of the concentration of hydrogen ions in solution in \si{\molar}. The \pOH{} of a solution is similarly defined.

Due to the laws of logarithm, the following relationship arises: \begin{equation}\p{\Kw} = \pH + \pOH\end{equation}

A solution is said to be \sldef{neutral} when \(\conc{H^+} = \conc{OH^-}\), \sldef{acidic} when \(\conc{H^+} > \conc{OH^-}\), and \sldef{basic} when \(\conc{H^+} < \conc{OH^-}\).

To calculate the \pH{} of a strong acid, find the \conc{H^+} and from that the \pH{}. When the concentration of protons from the acid is below \SI{10e-5}{\molar}, the hydrogen ions (\SI{10e-7}{\molar}) produced by water dissociation should be taken into account. This is done likewise for a base, and then calculating the \pH{} from the \pOH{}.

All acids have an acid dissociation constant \(\Ka\) that measures its strength, given by \begin{equation}\Ka = \frac{\conc{H^+}\conc{A^-}}{\conc{HA}}\end{equation} For strong acids, the \(\Ka\) is more than that of the hydronium ion i.e. more than about \num{54.95}. The larger the \(\Ka\), the stronger the acid.

Bases likewise have a base dissociation constant \(\Kb\), given by \begin{equation}\Kb = \frac{\conc{BH^+}\conc{OH^-}}{\conc{B}}\end{equation} The larger the \(\Kb\), the stronger the base.

The concentration of \slch{H^+} produced by a weak acid and of \slch{OH^-} by a weak base in solution are approximated by \begin{align}\conc{H^+} &\approx \sqrt{\conc{HA} \times K_{a}}\\\conc{OH^-} &\approx \sqrt{\conc{B} \times K_{b}}\end{align}

Considering the expressions for \(\Kw\), \(\Ka\) and \(\Kb\), the following relationship can be derived: \begin{equation}\Kw = \Ka\Kb\end{equation} where \(\Ka\) is the dissociation constant of an acid and \(\Kb\) is that of its conjugate base, or vice versa.

The conjugate base of a strong acid can be considered to be neutral, while that of a weak acid will be weakly basic as well. The same applies to bases.
\section{Buffers}
A \sldef{buffer solution} is one that is capable of maintaining a fairly constant \pH{} when small amounts of acid or base are added to it.

An \sldef{acidic buffer} contains a weak acid and the salt of its conjugate base, while a \sldef{basic buffer} contains a weak base and the salt of its conjugate acid.

A buffer solution is most effective in resisting \pH{} change when the concentration of the acid or base equals the concentration of its conjugate. This is known as \sldef{maximum buffer capacity}, which occurs when \[\pH = \pKa \vee \pOH = \pKb\]

The \pH{} of a buffer system can be calculated by \begin{align}\pH &= \pKa + \lg\frac{\conc{A^-}}{\conc{HA}}\\\pOH &= \pKb + \lg\frac{\conc{BH^+}}{\conc{B}}\end{align}

A buffer works by removing any \slch{H^+} or \slch{OH^-} added. When a small amount of acid, \slch{H^+}, is added, the added \slch{H^+} is removed as \ldots{}. \conc{H^+} is only slightly changed and so \pH{} remains fairly constant. When a small amount of base, \slch{OH^-}, is added, the added \slch{OH^-} is removed as \ldots{}. \conc{OH^-} is only slightly changed and so \pH{} remains fairly constant.

\pH{} in blood is maintained by the \slch{H2CO3}/\slch{HCO3^-} buffer pair.

When there is excess acid in the blood, \begin{equation}\begin{split}\ch{H^+ + HCO3^- -> H2CO3}\\\ch{H2CO3 -> CO2 + H2O}\end{split}\end{equation} Excess acid in the blood reacts with the hydrogencarbonate ion to form carbonic acid, which then decomposes into water and carbon dioxide, which is exhaled.

When there is too little acid present, \begin{equation}\ch{H2CO3 -> H^+ + HCO3^-}\end{equation} By Le Chatelier's principle, the equilibrium position shifts right, causing an increase in \conc{H^+}.
\section{Acid-base titrations}
An acid can be titrated against a base, or vice versa. The substance in the conical flask is titrated against the substance in the burette, or the substance in the burette is titrated into the substance in the flask.

The \sldef{equivalence point} of a titration is the theoretical point when stoichiometric amounts of the acid and base have reacted. There is usually a rapid \pH{} change over this point.

The \sldef{endpoint} of a titration is the point where the indicator used changes colour. Any indicator selected should have a \pH{} transition range that lies within the rapid change over the equivalence point.

A \sldef{titration curve} is simply a graph of \pH{} against volume of titrant added.

Most titrations have only one equivalence point. However, titrations involving weak acids or bases that are polyprotic will have the same number of equivalence points as the number of protons that the acid or base can donate or accept.

For a titration with a weak acid or base, at the each equivalence point, the main species contributing to \pH{} or \pOH{} is the conjugate base or acid of the previous main species (or the weak acid or base itself).

For a titration of a weak diprotic acid with a strong base, at the first equivalence point, \begin{equation}\pH \approx \frac{1}{2}(\pKa[1] + \pKa[2])\end{equation}

The endpoint of a weak base-strong acid or strong base-weak acid titration is acidic or basic respectively, due to salt hydrolysis causing \(\conc{H^+} > \conc{OH^-}\) or \(\conc{OH^-} > \conc{H^+}\) respectively.
\section{Salt solubility}
Soluble salts dissociate completely in solution into their constituent ions, but sparingly soluble salts dissociate only partially.

The \sldef{solubility product} is an equilibrium constant that is the product of the molar concentrations of the dissolved dissociated ions, each raised to their appropriate powers, in a saturated solution of the salt at a given temperature.

\sldef{Solubility} is a concentration term that refers to the maximum amount of solute that can be dissolved per \si{\deci\metre\cubed} of solvent to make a saturated solution at a given temperature.

\sldef{Common ion effect} occurs when the solubility of a sparingly soluble salt is decreased in the presence of a common ion from an external source.

When a common ion is added, due to common ion effect, by Le Chatelier's principle, equilibrium position shifts left to decrease the concentration of the common ion. The solubility of the salt will be decreased.

The \sldef{ionic product} is a product of the molar concentrations of the dissolved dissociated ions, each raised to their appropriate powers, in the solution at a given temperature. Its value may vary with situation.

When the ionic product of a salt is less than the salt's solubility product, no precipitation occurs. When the ionic product equals the solubility product, the solution is saturated; no precipitation occurs yet. When the ionic product is greater than the solubility product, precipitation of the salt occurs until the ionic product equals the solubility product.
\end{document}
