\documentclass[Chemistry.tex]{subfiles}
\begin{document}
\chapter{Inorganic Chemistry Summary of Reactions}
\begin{tabularx}{\textwidth}[c]{clXc}
\sltbcap{Reaction of period 3 elements with oxygen}{tb:a9.eox}
\toprule
& \sltbhdr{Equation} & \sltbhdr{Observations} & \textbf{Flame} \\
\midrule \endhead
\ch{Na} & \ch{2 Na\solid{} + 1/2 O2\gas{} -> Na2O\solid{}} & Burns very vigorously & Yellow \\
\midrule
\ch{Mg} & \ch{Mg\solid{} + 1/2 O2\gas{} -> MgO\solid{}} & Burns very vigorously & Bright white \\
\midrule
\ch{Al} & \ch{4 Al\solid{} + 3 O2\gas{} -> 2 Al2O3\solid{}} & Must be heated to \SI{800}{\celsius} due to the unreactive oxide layer & --- \\
\midrule
\ch{Si} & \ch{Si\solid{} + O2\gas{} -> SiO2\solid{}} & Reacts slowly with strong heat & --- \\
\midrule
\ch{P} & \begin{varwidth}[t]{0.4\textwidth}\ch{P4\solid{} + 3 O2\gas{} -> P4O6\solid{}}\\\ch{P4\solid{} + 5 O2\gas{} -> P4O10\solid{}}\end{varwidth} & Reacts vigorously, forming a dense white fume of \ch{P4O10} & Brilliant yellow \\
\midrule
\ch{S} & \begin{varwidth}[t]{0.4\textwidth}\ch{S\solid{} + O2\gas{} -> SO2\gas{}}\\\ch{SO2\gas{} + 1/2 O2\gas{}~(excess) -> SO3\gas{}}\end{varwidth} & Burns slowly & Blue \\
\bottomrule
\end{tabularx}
%
\begin{longtable}[c]{cll}
\sltbcap{Reaction of period 3 elements with chlorine}{tb:a9.ecl}
\toprule
& \sltbhdr{Chloride} & \sltbhdr{Observations} \\
\midrule\endhead
\ch{Na} & \ch{NaCl\solid{}} & Reacts very vigorously \\
\midrule
\ch{Mg} & \ch{MgCl2\solid{}} & Reacts vigorously \\
\midrule
\ch{Al} & \ch{AlCl3\solid{}} & Reacts vigorously; dimerises to \ch{Al2Cl6} \\
\midrule
\ch{Si} & \ch{SiCl4\lqd{}} & Reacts slowly \\
\midrule
\ch{P4\solid{}} & \ch{PCl3\lqd{}} & In limited \ch{Cl2}; reacts slowly \\
\ch{P4\solid{}} & \ch{PCl5\solid{}} & In excess \ch{Cl2}; reacts slowly \\
\midrule
\ch{S} & \ch{S2Cl2\lqd{}} & Reacts slowly \\
\bottomrule
\end{longtable}
%
\begin{longtable}[c]{cll}
\sltbcap{Reaction of period 3 elements with water}{tb:a9.ew}
\toprule
 & \sltbhdr{Equation} & \textbf{Observations} \\
\midrule\endhead
\ch{Na} & \ch{2 Na\solid{} + 2 H2O\lqd{} -> 2 NaOH\aq{} + H2\gas{}} & Reacts very vigorously \\
\midrule
\ch{Mg} & \ch{Mg\solid{} + H2O\gas{} -> MgO\solid{} + H2\gas{}} & Reacts with steam only \\
\midrule
\ch{Al} & \ch{2 Al\solid{} + 3 H2O\gas{} -> Al2O3\solid{} + 3 H2\gas{}} & Reacts with steam only \\
\midrule
\ch{Cl2} & \ch{Cl2\gas{} + H2O\lqd{} -> HClO\aq{} + HCl\gas{}} & Hydrolyses to \pH 2 solution \\
\midrule
Rest & \multicolumn{2}{l}{Does not react} \\
\bottomrule
\end{longtable}
%
\clearpage
\begin{tabularx}{\textwidth}[c]{clXcc}
\sltbcap{Reaction of period 3 oxides with water}{tb:a9.oxw}
\toprule
& \sltbhdr{Equation} & \sltbhdr{Remarks} & \(\pH\) & \textbf{UI} \\
\midrule\endhead
\ch{Na2O} & \ch{Na2O\solid{} + H2O\lqd{} -> 2 NaOH\aq{}} & Reacts vigorously & \num{13} & Violet \\
\midrule
\ch{MgO} & \ch{MgO\solid{} + H2O\lqd{} <=> Mg(OH)2\aq{}} & Reacts less vigorously and dissolves sparingly as its lattice energy is high, leading to a high enthalpy of solution. & \num{9} & Blue \\
\midrule
\ch{Al2O3} & \multicolumn{4}{l}{Does not react; too much energy required to cause detachment of ions from the lattice structure} \\
\midrule
\ch{SiO2} & \multicolumn{4}{l}{Does not react; too much energy required to break its very stable giant molecular structure} \\
\midrule
\ch{P4O6} & \ch{P4O6\solid{} + 6 H2O\lqd{} -> 4 H3PO3\aq{}} & & \num{2} & Red \\
\ch{P4O10} &\ch{P4O10\solid{} + 6 H2O\lqd{} -> 4 H3PO4\aq{}} \\
\cmidrule{1-3}
\ch{SO2} &\ch{SO2\gas{} + H2O\lqd{} -> H2SO3\aq{}} \\
\ch{SO3} & \ch{SO3 + H2O\lqd{} -> H2SO4\aq{}} \\
\cmidrule{1-3}
\ch{Cl2O7} & \ch{Cl2O7\aq{} + H2O\lqd{} -> 2 HClO4\aq{}} \\
\bottomrule
\end{tabularx}
%
\begin{longtable}[c]{cll}
\sltbcap{Reaction of period 3 oxides with acid and base}{tb:a9.oa}
\toprule
& \sltbhdr{Equation} & \sltbhdr{Remarks} \\
\midrule\endhead
\ch{Na2O} & \ch{Na2O\solid{} + 2 H^+\aq{} -> 2 Na^+\aq{} + H2O\lqd{}} & \\
\midrule
\ch{MgO} & \ch{MgO\solid{} + 2 H^+\aq{} -> Mg^{2+}\aq{} + H2O\lqd{}} & \\
\midrule
\ch{Al2O3} & \ch{Al2O3\solid{} + 6 H^+\aq{} -> 2 Al^{3+}\aq{} + 3 H2O\lqd{}} & With acid \\
\ch{Al2O3} & \ch{Al2O3\solid{} + 2 OH^-\aq{} + 3 H2O\lqd{} -> 2 Al(OH)4^{-}\aq{}} & With base \\
\midrule
\ch{SiO2} & \ch{SiO2\solid{} + 2 OH^{-}\aq{} -> SiO3^{2-}\aq{} + H2O\lqd{}} & \\
\midrule
\ch{P4O6} & \ch{P4O6\solid{} + 8 OH^{-}\aq{} -> 4 HPO3^{2-}\aq{} + 2 H2O\lqd{}} & \\
\ch{P4O10} & \ch{P4O10\solid{} + 12 OH^{-}\aq{} -> 4 HPO4^{3-}\aq{} + 6 H2O\lqd{}} & \\
\midrule
\ch{SO2} & \ch{SO2\gas{} + 2 OH^{-}\aq{} -> SO3^{2-}\aq{} + H2O\lqd{}} & \\
\ch{SO3} & \ch{SO3\gas{} + 2 OH^{-}\aq{} -> SO4^{2-}\aq{} + H2O \lqd{}} & \\
\midrule
\ch{Cl2O7} & \ch{Cl2O7\aq{} + 2 OH^{-}\aq{} -> 2 ClO4^{-}\aq{} + H2O\lqd{}} & \\
\bottomrule
\end{longtable}
%
\begin{tabularx}{\textwidth}[c]{clXcc}
\sltbcap{Reaction of period 3 chlorides with water}{tb:a9.clw}
\toprule
& \sltbhdr{Equation} & \sltbhdr{Remarks} & \(\pH\) & \textbf{UI} \\
\midrule\endhead
\ch{NaCl} & \ch{NaCl\solid{} -> Na^+\aq{} + Cl^-\aq{}} & Undergoes hydration & 7 & Green \\
\midrule
\ch{MgCl2} & \multicolumn{2}{l}{\begin{varwidth}[t]{0.7\textwidth}\ch{MgCl2\solid{} + 6 H2O\lqd{} -> {[}Mg(H2O)6{]}^{2+}\aq{} + 2 Cl^{-}\aq{}}\\\ch{{[}Mg(H2O)6{]}^{2+}\aq{} + H2O\lqd{} <=> {[}Mg(H2O)5(OH){]}^{+}\aq{} + H3O^{+}\aq{}}\end{varwidth}} & \num{6.5} & Orange \\\addlinespace
\multicolumn{5}{p{0.9\textwidth}}{Undergoes hydration and then slight hydrolysis to form a slightly acidic solution, as the relatively high charge density of the hydrated \slch{Mg^{2+}} ion polarises the electron cloud of one of the surrounding water molecules, weakening and breaking the \slch{O-H} bond, releasing a proton.} \\
\midrule
\ch{Al2Cl6} & \multicolumn{2}{l}{\begin{varwidth}[t]{0.7\textwidth}\ch{AlCl3\solid{} + 6 H2O\lqd{} -> {[}Al(H2O)6{]}^{3+}\aq{} + 3 Cl^{-}\aq{}}\\\ch{{[}Al(H2O)6{]}^{3+}\aq{} + H2O\lqd{} <=> {[}Al(H2O)5(OH){]}^{2+}\aq{} + H3O^{+}\aq{}}\end{varwidth}} & \num{3} & Orange \\\addlinespace
\multicolumn{5}{p{0.9\textwidth}}{Undergoes hydration and hydrolysis to form an acidic solution, for reasons similar to \slch{MgCl2}.} \\
\midrule
\ch{SiCl4} & \ch{SiCl4\lqd{} + 2 H2O\lqd{} -> SiO2\solid{} + 4 HCl\gas{}} & Hydrolyses completely & \num{2} & Red \\
\cmidrule{1-3}
\ch{PCl3} & \ch{PCl3\lqd{} + H2O\lqd{} -> H3PO3\aq{} + 3 HCl\gas{}} \\
\ch{PCl5} & \ch{PCl5\solid{} + 4 H2O\lqd{}~(excess) -> H3PO4\aq{} + 5 HCl\gas{}} & (hot) \\
\ch{PCl5} & \ch{PCl5\solid{} + H2O\lqd{} -> POCl3\aq{} + 2 HCl\aq{}} & (cold) \\
\ch{POCl3} & \ch{POCl3\aq{} + 3 H2O\lqd{} -> H3PO4\aq{} + 3 HCl\aq{}} \\
\cmidrule{1-3}
\ch{S2Cl2} & \ch{2 S2Cl2\lqd{} + 2 H2O\lqd{} -> 3 S\solid{} + SO2\gas{} + 4 HCl\gas{}} \\
\cmidrule{1-3}
\ch{Cl2} & \ch{Cl2\gas{} + H2O\lqd{} -> HClO\aq{} + HCl\gas{}} \\
\bottomrule
\end{tabularx}
%
\clearpage
\begin{longtable}[c]{cccccccc}
\sltbcap{Reactions of group II elements and oxides}{tb:a9.gii}
\toprule
\multicolumn{2}{c}{\textbf{Element}} & \multicolumn{2}{c}{\textbf{Reaction with water}} & \multicolumn{4}{c}{\textbf{Oxide}}\\
\cmidrule(r){1-2} \cmidrule(lr){3-4} \cmidrule(l){5-8}
& \textbf{Flame} & \textbf{Cold} & \textbf{Steam} & & \textbf{Solubility} & \(\pH\) & \textbf{UI}\\
\cmidrule(r){1-2} \cmidrule(lr){3-4} \cmidrule(l){5-8}\endhead
\slch{Be} & --- & --- & --- & \ch{BeO} & \multicolumn{3}{c}{Insoluble}\\
\slch{Mg} & brilliant white & --- & Forms oxide & \slch{MgO} & Slightly & \num{9} & Blue\\
\slch{Ca} & red & \multicolumn{2}{c}{Forms hydroxide} & \slch{CaO} & Yes & \numrange{10}{13} & Violet\\
\slch{Sr} & crimson & \multicolumn{2}{c}{Forms hydroxide} & \slch{SrO} & Yes & \numrange{10}{13} & Violet\\
\slch{Ba} & green & \multicolumn{2}{c}{Forms hydroxide} & \slch{BaO} & Yes & \numrange{10}{13} & Violet\\
\bottomrule
\end{longtable}
%
\begin{longtable}[c]{cccccc}
\sltbcap{Colours of group VII elements}{tb:a9.g7.col}
\toprule
& \multicolumn{5}{c}{\textbf{Colour in state}}\\
\cmidrule{2-6}
& \textbf{Gas} & \textbf{Liquid} & \textbf{Solid} & \textbf{Aqueous} & \textbf{Organic} \\
\ch{Cl2} & \multicolumn{2}{c}{Greenish-yellow} & --- & Greenish-yellow & Yellow \\
\ch{Br2} & \multicolumn{2}{c}{Reddish-brown} & --- & Yellow & Orange \\
\ch{I2} & Violet & --- & Black & Brown & Violet \\
\bottomrule
\end{longtable}
%
\begin{tabularx}{\textwidth}[c]{clX}
\sltbcap{Reactions of group VII elements with hydrogen}{tb:a9.g7.eh}
\toprule
& \sltbhdr{Equation} & \sltbhdr{Observations} \\
\midrule\endhead
\ch{F2} & \ch{H2\gas{} + F2\gas{} -> 2 HF\gas{}} & Explosive even in the dark \\
\ch{Cl2} & \ch{H2\gas{} + Cl2\gas{} -> 2 HCl\gas{}} & Explosive in sunlight; does not react at r.t.p. or in the dark \\
\ch{Br2} & \ch{H2\gas{} + Br2\gas{} -> 2 HBr\gas{}} & Heat and \ch{Pt} catalyst \\
\ch{I2} & \ch{H2\gas{} + I2\gas{} -> 2 HI\gas{}} & Reacts reversibly at \SI{400}{\celsius} with \ch{Pt} catalyst \\
\bottomrule
\end{tabularx}
%
\begin{tabularx}{\textwidth}[c]{clX}
\sltbcap{Reactions of group VII elements with \ch{NaOH\aq{}}}{tb:a9.g7.enaoh}
\toprule
& \sltbhdr{Equation} & \sltbhdr{Remarks} \\
\midrule\endhead
\ch{X2} & \ch{2 OH^-\aq{} + X2\aq{} -> X^-\aq{} + XO^-\aq{} + H2O\lqd{}} & R.t.p. dilute \ch{NaOH} and \ch{Cl2} or \SI{0}{\celsius} dilute \ch{NaOH} and \ch{Br2} \\
\ch{XO^-} & \ch{3 XO^-\aq{} -> 2 X^-\aq{} + XO3^-\aq{}} & On warming \\
\midrule
\ch{X2} & \ch{6 OH^-\aq{} + 3 X2\aq{} -> 5 X^-\aq{} + XO3^-\aq{} + 3 H2O\lqd{}} & Hot concentrated \ch{NaOH} (\SI{70}{\celsius}) and \ch{Cl2} \textbf{or} \ch{Br2} or \ch{I2} \\
\bottomrule
\end{tabularx}
%
\begin{tabularx}{\textwidth}[c]{clX}
\sltbcap{Reactions of group VII elements with concentrated \ch{H2SO4}}{tb:a9.g7.eh2so4}
\toprule
& \sltbhdr{Equation} & \sltbhdr{Remarks} \\
\midrule\endhead
\ch{F2}, \ch{Cl2} & \ch{NaX\solid{} + H2SO4\lqd{} -> HX\gas{} + NaHSO4\solid{}} & \ch{HX} is not further oxidised by \ch{H2SO4} as the latter is not powerful enough an oxidising agent \\
\midrule
\ch{Br2} & \begin{varwidth}[t]{0.5\textwidth}\ch{NaBr\solid{} + H2SO4\lqd{} -> HBr\gas{} + NaHSO4\solid{}}\\\ch{2 HBr\gas{} + H2SO4\lqd{} -> Br2\gas{} + SO2\gas{} + 2 H2O\lqd{}}\end{varwidth} & \ch{HBr} forms white fumes; \ch{SO2} is pungent \\
\midrule
\ch{I2} & \begin{varwidth}[t]{0.5\textwidth}\ch{NaI\solid{} + H2SO4\lqd{} -> HI\gas{} + NaHSO4\solid{}}\\\ch{8 HI\gas{} + H2SO4\lqd{} -> 4 I2\gas{} + H2S\gas{} + 4 H2O\lqd{}}\end{varwidth} & \ch{HI} forms white fumes; \ch{H2S} is pungent \\
\bottomrule
\end{tabularx}
%
\clearpage
\begin{longtable}[c]{llll}
\sltbcap{Colours of transition metal compounds and complexes}{tb:a9.tm.col}
\toprule
\sltbhdr{Ion} & \multicolumn{3}{c}{\textbf{Species and colour}} \\
\midrule\endhead
\ch{V}(II) & \ch{[V(H2O)6]^{2+}}: violet \\
\ch{V}(III) & \ch{[V(H2O)6]^{3+}}: green \\
\ch{V}(IV) & \ch{[VO(H2O)5]^{2+}}: blue \\
\ch{V}(V) & \ch{[VO2(H2O)4]^{+}}: yellow \\
\midrule
\ch{Cr}(II) & \ch{[Cr(H2O)6]^{2+}}: blue \\
\ch{Cr}(III) & \ch{[Cr(H2O)6]^{3+}}: green & \ch{[Cr(OH)6]^{3-}}: deep green & \ch{[Cr(NH3)6]^{3+}}: purple \\
\ch{Cr}(VI) & \ch{CrO4^{2-}}: yellow & \ch{Cr2O7^{2-}}: orange \\
\midrule
\ch{Mn}(II) & \ch{[Mn(H2O)6]^{2+}}: pink/colourless \\
\ch{Mn}(III) & \ch{[Mn(H2O)6]^{3+}}: red \\
\ch{Mn}(IV) & \ch{MnO2}: brown solid \\
\ch{Mn}(VI) & \ch{MnO4^{2-}}: green \\
\ch{Mn}(VII) & \ch{MnO4^-}: purple \\
\midrule
\ch{Fe}(II) & \ch{[Fe(H2O)6]^{2+}}: pale green & \ch{[Fe(CN)6]^{4-}}: yellow \\
\ch{Fe}(III) & \ch{[Fe(H2O)6]^{3+}}: yellow & \ch{[Fe(CN)6]^{3-}}: orange-red & \ch{[Fe(H2O)5(SCN)]^{2+}}: blood red \\
\midrule
\ch{Co}(II) & \ch{[Co(H2O)6]^{2+}}: pink & \ch{[Co(NH3)6]^{2+}}: pale brown & \ch{[CoCl4]^{2-}}: blue \\
\ch{Co}(III) & \ch{[Co(H2O)6]^{3+}}: dark brown \\
\midrule
\ch{Ni}(II) & \ch{[Ni(H2O)6]^{2+}}: green & \ch{[Ni(NH3)6]^{2+}}: blue & \ch{[Ni(CN)6]^{4-}}: yellow \\
\midrule
\ch{Cu}(I) & \ch{Cu2O}: reddish-brown solid \\
\ch{Cu}(II) & \ch{[Cu(H2O)6]^{2+}}: blue & \ch{[Cu(NH3)4(H2O)2]^{2+}}: dark blue & \ch{[CuCl4]^{2-}}: yellow \\
\midrule
\ch{Ag}(I) & \ch{[Ag(H2O)2]^{+}}: colourless & \ch{[Ag(NH3)2]^{+}}: colourless \\
\bottomrule
\end{longtable}
%
\begin{longtable}[c]{ccccc}
\sltbcap{Transition metal precipitates soluble when excess ligand added}{tb:a9.tm.ex}
\toprule
\sltbhdr{Precipitate} & \sltbhdr{Colour} & \sltbhdr{Soluble in excess of} & \sltbhdr{Complex} & \sltbhdr{Colour} \\
\midrule\endhead
\ch{Cr(OH)3} & Green & \ch{NaOH\aq{}} & \ch{[Cr(OH)6]^{3-}} & Deep green \\
\midrule
\ch{Zn(OH)2} & White & \ch{NaOH\aq{}} & \ch{[Zn(OH)4]^{2-}} & Colourless \\
\ch{Zn(OH)2} & White & \ch{NH3\aq{}} & \ch{[Zn(NH3)4]^{2+}} & Colourless \\
\midrule
\ch{Cu(OH)2} & Blue & \ch{NH3\aq{}} & \ch{[Cu(NH3)4(H2O)2]^{2+}} & Deep blue \\
\midrule
\ch{Co(OH)2} & Blue & \ch{NH3\aq{}} & \ch{[Co(NH3)6]^{2+}} & Pale brown \\
\midrule
\ch{Ni(OH)2} & Green & \ch{NH3\aq{}} & \ch{[Ni(NH3)6]^{2+}} & Blue \\
\bottomrule
\end{longtable}
\end{document}
