\cleardoublepage
\chapter{Annex: Experimental Procedures}
\section{Titration}
\subsection{Preparing standard solution of X}
\restartlist{enumerate}
\begin{enumerate}

\item
  Weigh \(m\) \si{\gram} of \emph{(the substance)} in a
  weighing bottle using an electronic balance. Record the total mass of
  \emph{(the substance)} and the weighing bottle.
\item
  Transfer the weighed sample to a \SI{250}{\centi\metre\cubed} beaker.
\item
  Reweigh the weighing bottle to account for any residual solid in the
  weighing bottle. Record the total mass of the residue and the weighing
  bottle.
\item
  Using a \SI{10}{\centi\metre\cubed} measuring cylinder, measure
  \emph{(volume)} \si{\centi\metre\cubed} of \emph{(acid)} and
  transfer this measured acid into the \SI{250}{\centi\metre\cubed} beaker.
\item
  Using \SI{100}{\centi\metre\cubed} of deionised water, dissolve the
  sample completely in the beaker.
\item
  Transfer the resultant solution into a \SI{250}{\centi\metre\cubed}
  graduated volumetric flask
\item
  Top the volumetric flask up to the graduated mark with deionised
  water. Shake the flask vigorously.
\end{enumerate}

\subsection{Titration}

\begin{enumerate}

\item
  Pipette \SI{25.0}{\centi\metre\cubed} of \emph{(analyte)} from the
  volumetric flask into a conical flask, scaling the solution down by a
  factor of \num{10}.
\item
  Titrate the \emph{(analyte)} with \emph{(titrant)} from
  a burette until the solution in the conical flask turns
  \emph{(colour change)}. Record the initial and final burette
  readings.
\item
  Repeat the titration to obtain results consistent within \SI{\pm0.10}{\centi\metre\cubed}.
\end{enumerate}

\subsection{Results}
\subsubsection{Mass readings}
\begin{longtable}[c]{@{}ll@{}}
\toprule
Mass of weighing bottle and (the substance)/\si{\gram} & \(m_1\)\tabularnewline
Mass of weighing bottle and residue/\si{\gram} & \(m_2\)\tabularnewline
Mass of (the substance) used/\si{\gram} & \(m = m_1 - m_2\)\tabularnewline
\bottomrule
\end{longtable}
\subsubsection{Burette readings (titration)}
\begin{longtable}[c]{@{}llll@{}}
\toprule
Titration number & 1 & 2 & \ldots{}\tabularnewline\midrule
\endhead
Final burette reading/\si{\centi\metre\cubed} & & & \ldots{}\tabularnewline
Initial burette reading/\si{\centi\metre\cubed} & & & \ldots{}\tabularnewline
Volume of (titrant) used/\si{\centi\metre\cubed} & \(v_1\) & \(v_2\) & \ldots{}\tabularnewline
Best titration results & \slcheckmark & \slcheckmark &\tabularnewline
\bottomrule
\end{longtable}
(\(v_1\) and \(v_2\) must be within
\SI{\pm0.10}{\centi\metre\cubed} of each other)

\SI{25.0}{\centi\metre\cubed} of \emph{(analyte)} required
\(\frac{v_{1} + v_{2}}{2}\) \si{\centi\metre\cubed} of
\emph{(titrant)} for complete reaction. The end-point colour
change was \emph{(colour change)}.

\section{Thermochemistry: Enthalpy change of solution/neutralisation}
\restartlist{enumerate}
\begin{enumerate}

\item
  Measure \SI{100}{\centi\metre\cubed} of \emph{(solvent/reagent 1)}
  using a \SI{100}{\centi\metre\cubed} measuring cylinder and transfer it
  into a \SI{200}{\centi\metre\cubed} polystyrene cup.
\item
  Measure and record the initial temperature of the solution in the cup
  with a thermometer.
\item
  Weigh out \emph{(mass)} \si{\gram}/Measure \SI{100}{\centi\metre\cubed} of
  \emph{(solute/reagent 2)} using an electronic balance in a
  clean and dry weighing bottle/\SI{100}{\centi\metre\cubed} measuring
  cylinder. Record the total mass of the \emph{(solute)} and the
  weighing bottle.
\item
  Transfer the \emph{(solute/reagent 2)} into the cup. Stir and
  record the highest temperature attained, using a thermometer.
\item
  Reweigh the weighing bottle to account for any residual solid. Record
  the total mass of the residue and the weighing bottle.
\end{enumerate}

Include suitable tables.

\section{Thermochemistry: Enthalpy change of combustion}
\restartlist{enumerate}
\begin{enumerate}

\item
  Measure \SI{100}{\centi\metre\cubed} of water using a \SI{100}{\centi\metre\cubed} measuring cylinder and transfer it a calorimeter
  with a lid.
\item
  Weigh and record the initial mass of the spirit lamp containing
  \emph{(fuel)} using an electronic balance.
\item
  Monitor the temperature of the water in the calorimeter using a
  thermometer for \SI{2}{\minute}, and record the average temperature as the
  initial temperature.
\item
  Light the spirit lamp with a match. Allow \emph{(fuel)} to burn
  until the temperature rises by about \SIrange{7}{10}{\celsius}.
\item
  Extinguish the lamp. Continue observing the rise in temperature.
  Record the highest temperature reached.
\item
  Reweigh and record the final mass of the spirit lamp containing
  \emph{(fuel)} using an electronic balance.
\end{enumerate}

Include suitable tables.
